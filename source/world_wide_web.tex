\section{World Wide Web}
    Il \textbf{World Wide Web} (abbreviato \textbf{WWW} o \textbf{Web}) è un’architettura che consente di accedere a
    documenti distribuiti su un vasto numero di macchine nell’Internet globale.\\

    I documenti sono tecnicamente chiamati \textbf{pagine Web} (spesso dette solo \textbf{pagine} per brevità) e
    possono essere di 2 tipi: \textbf{statiche} (la pagina ha sempre lo stesso contenuto ad ogni
    visualizzazione) o \textbf{dinamiche} (viene generato il contenuto della pagina ad ogni richiesta,
    tramite un linguaggio di scripting come il PHP). Ogni pagina è localizzata da uno o più \textbf{URL
    (Uniform Resource Locator)}, una stringa composta di 3 parti:

        \begin{itemize}
            \item Il nome del protocollo (es: http, https);
            \item Il nome DNS della macchina su cui è sita la risorsa (es: \url{www.unipi.it});
            \item La posizione locale sul server della risorsa richiesta (es: index.php).
        \end{itemize}

    Un esempio di URL è quindi: \url{https://www.unipi.it/index.php}. Si noti che l’URL è un tipo
    specifico di \textbf{URI (Uniform Resource Identifier)}. La differenza sta nel fatto che l’URI identifica
    una risorsa, mentre l’URL la localizza. Si noti che localizzare una risorsa vuol dire
    conseguentemente anche identificarla, ma non è sempre vero il contrario, per cui i due termini
    non possono essere usati come sinonimi: ad esempio, l’ISBN di un libro x è un URI (in quanto
    univoco), ma non localizza il libro, per cui non è anche un URL.\\
    
    Ogni pagina può contenere collegamenti ad altre pagine situate ovunque nel mondo. L’insieme
    delle pagine collegate tra loro è detto \textbf{hypertext (ipertesto)}, ed il collegamento cliccabile che
    reindirizza ad altre pagine è detto \textbf{hyperlink} o \textbf{link (collegamento ipertestuale)}.\\

    Il software per l’ottenimento, il rendering e la navigazione di pagine è detto \textbf{browser}.

\subsection{HTTP}
    Il protocollo di trasferimento usato sul Web è \textbf{HTTP (HyperText Transfer Protocol)}, un
    protocollo \textbf{stateless} (ovvero non salva informazioni sulla sessione) che permette la
    realizzazione di sistemi informativi distribuiti, collaborativi ed ipermediali (ossia composti da
    multimedialità distribuita nella rete ed acceduta mediante hyperlink).\\

    Il metodo utilizzato da un browser per contattare un server prevede di stabilire una
    connessione TCP di solito alla porta 80. Da HTTP/1.1 sono supportate le \textbf{connessioni
    persistenti}, che permettono di inviare più di una richiesta al server con la stessa connessione.\\

    Una \textbf{HTTP request} (richiesta HTTP) è composta da:

        \begin{itemize}
            \item \textbf{Request line}: singola stringa composta da un metodo (\textbf{GET, POST, HEAD, PUT,
            DELETE, TRACE, OPTIONS} o \textbf{CONNECT}), un URL e la versione del protocollo.
            Un esempio è: GET url_del_file HTTP/1.1;
            \item \textbf{Header}: una o più stringhe contenenti ulteriori informazioni, seguiti da una riga vuota.
            Tra i più comuni vi sono:

            \begin{itemize}
                \item \textbf{Host}: indica il nome del server a cui si riferisce l’URL;
                \item \textbf{User-agent}: identifica il tipo di client: tipo di browser, versione, ecc.;
                \item \textbf{Cookie}: utilizzati dalle applicazioni web per archiviare e recuperare
                informazioni a lungo termine sul lato client.
            \end{itemize}

            \item \textbf{Body}: dati opzionali che si vogliono trasmettere.
        \end{itemize}

    Una \textbf{HTTP response} (risposta HTTP) è composta da:

        \begin{itemize}
            \item \textbf{Status line}: una riga contente uno status code ed un messaggio esplicativo. Gli status
            code sono dei numeri a 3 cifre che la cui prima cifra specifica una delle 5 categorie:

            \begin{itemize}
                \item 1xx Informational;
                \item 2xx Success;
                \item 3xx Redirection;
                \item 4xx Client Error;
                \item 5xx Server Error.
            \end{itemize}

            \item \textbf{Header}: una o più stringhe contenenti ulteriori informazioni, seguiti da una riga vuota.
            Tra i più comuni vi sono:

            \begin{itemize}
                \item \textbf{Content-Type}: contiene il tipo MIME della pagina;
                \item \textbf{Server}: contiene informazioni sul server;
                \item \textbf{Set-Cookie}: contiene un cookie che il server vuole che il client salvi.
            \end{itemize}

            \item \textbf{Body}: dati opzionali che si vogliono restituire.
        \end{itemize}

    Vi sono state varie versioni di HTTP. Di seguito vengono analizzati in breve i cambiamenti
    principali tra le versioni:

        \begin{itemize}
            \item \textbf{HTTP/0.9}: semplice protocollo per il trasferimento di dati grezzi sulla rete Internet,
            prima di esso il protocollo di riferimento per tali scopi era FTP;
            \item \textbf{HTTP/1.0}: pur consentendo il trasferimento di messaggi di tipo MIME, non era adatto
            a supportare la crescita esponenziale del Web;
            \item \textbf{HTTP/1.1}: versione consolidata del protocollo, utilizzata per ben 15 anni;
            \item \textbf{HTTP/2.0}: nuovo standard basato su un protocollo sviluppato da Google chiamato
            SPDY/2, che va a migliorare di molto le prestazioni mantenendo la retrocompatibilità
            per applicazioni già sviluppate;
        \end{itemize}

    \subsection{Pagine Web dinamiche}
        Come anticipato, esistono due categorie di pagine web: statiche e dinamiche. Per richiedere una
        pagina statica il procedimento è banale: si effettua una HTTP request con il nome del file e si
        ottiene una HTTP response con il documento richiesto. Per le pagine dinamiche il procedimento
        è un po’ più articolato poiché, una volta ricevuta la richiesta HTTP, il documento va generato su
        richiesta, con possibilità che ciò avvenga sia lato client che server.\\

        Per gestire le pagine Web dinamiche \textbf{lato server}, è possibile utilizzare un sistema chiamato \textbf{CGI
        (Common Gateway Interface)}. È un’interfaccia standardizzata che consente ai server Web di
        comunicare con script e programmi di back-end che possono accettare l’input e generare
        pagine HTML in risposta. Un linguaggio usato per la scrittura di CGI è \textbf{Perl}. È anche possibile
        \textbf{incorporare piccoli script nelle pagine HTML} stesse (ad esempio con il \textbf{PHP}), e farli eseguire
        al momento della generazione.\\

        Il processo, che inizia alla ricezione di una HTTP request e termina all’invio di una HTTP
        response è detto \textbf{round trip}, ed essendo HTTP un protocollo stateless, i linguaggi di scripting
        server-side mettono a disposizione apposite variabili dette \textbf{variabili di sessione}, per poter
        salvare dei dati tra una connessione e l’altra.\\

        Per gestire le pagine Web dinamiche \textbf{lato client}, è necessario allo stesso modo incorporare
        degli script nelle pagine HTML. Il linguaggio di script più popolare per il lato client è \textbf{JavaScript}.