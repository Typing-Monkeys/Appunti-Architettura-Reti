\section{Sicurezza delle reti}

    Tra gli attributi dell’oggetto dell’analisi di sicurezza informatica, rientra la seguente triade:

    \begin{itemize}
        \item \textbf{Confidenzialità}: le informazioni possono essere \textbf{lette} solo dagli aventi diritto;
        \item \textbf{Integrità}: le informazioni possono essere \textbf{modificate} solo dagli aventi diritto;
        \item \textbf{Disponibilità}: le informazioni possono essere \textbf{accedute} al momento del bisogno.
    \end{itemize}

    Questa triade prende il nome di CIA, dalle iniziali inglesi dei 3 termini.\\
    In seguito, sono state definite 2 nuove proprietà:

    \begin{itemize}
        \item \textbf{Accountability}: ogni utente e azione devono poter essere identificati;
        \item \textbf{Auditability}: l’efficacia dei meccanismi scelti deve poter essere verificata.
    \end{itemize}

    Nell’ambito della sicurezza informatica, è inoltre comune usare i seguenti termini:

    \begin{itemize}
        \item \textbf{Privacy}: il diritto che ha ogni individuo di determinare quando, quanto, come e a chi
        comunicare le informazioni che riguardano sé stesso;
        \item \textbf{Non ripudio}: fornire l’evidenza irrefutabile che una certa azione è stata compiuta da 
        una determinata persona e non da altre o che un certo messaggio è stato spedito da un
        determinato mittente (firma digitale);
        \item \textbf{Autenticazione}: fornire la prova che l’utente è esattamente chi dice di essere (da non
        confondere con l’identificazione dell’accountability);
        \item \textbf{Tracciabilità}: capacità di tracciare le azioni compiute dagli utenti sulle risorse;
        \item \textbf{Forensics}: capacità di provare che determinati attacchi hanno avuto luogo.
    \end{itemize}

    Definiamo inoltre vulnerabilità un difetto di una componente di un sistema informatico,
    classificandola in 3 tipi principali:

    \begin{itemize}
        \item \textbf{Procedurali}: difetto nella modalità con cui si opera;
        \item \textbf{Organizzative}: difetto nelle persone che operano;
        \item Degli \textbf{strumenti informatici}: difetto nell’hardware/software utilizzato. Per questo tipo
        di vulnerabilità, esistono 3 sottotipi: \begin{itemize}
            \item \textbf{Specifica}: difetto nella \textbf{progettazione} di un componente: ad esempio quando un
            componente è, inutilmente, più generale del necessario;
            \item \textbf{Implementativa}: difetto nella \textbf{realizzazione} di un componente: ad esempio la
            mancanza di controlli sugli input;
            \item \textbf{Strutturale}: difetto che nasce dalla \textbf{combinazione} di uno o più componenti del
            sistema.
        \end{itemize}
    \end{itemize}

    Sfruttando una vulnerabilità, è possibile effettuare un attacco informatico. In generale, le varie
    fasi sono:

    \begin{itemize}
        \item Raccolta di informazioni;
        \item Individuazione delle vulnerabilità;
        \item Ricerca o costruzione di un programma (detto \textbf{exploit}) che sfrutti le vulnerabilità;
        \item Esecuzione dell’exploit;
        \item Installazione di strumenti per il controllo;
        \item Cancellazione delle tracce;
        \item Accesso ad un sottoinsieme di informazioni.
    \end{itemize}

    Alcune delle categorie dei possibili attacchi che possono andare a violare l’affidabilità di un
    server Web sono:

    \begin{itemize}
        \item \textbf{Estensibilità del server}: attacchi ai servizi offerti dal server. Ad esempio, potersi
        connettere ad un database via CGI;
        \item \textbf{Estensibilità del browser}: attacchi ai servizi fruibili da browser. Ad esempio, alcuni usi
        di ActiveX, Java, Javascript, VBSript possono compromettere la sicurezza;
        \item \textbf{Distruzione del servizio}: attacchi mirati a bloccare l’erogazione di un servizio. Un
        esempio sono gli attacchi DoS;
        \item \textbf{Supporto complicato}: attacchi mirati alle vulnerabilità dei protocolli di basso livello,
        alcune volte necessari per un funzionamento corretto del sistema.
    \end{itemize}

    Per poter descrivere lo stato di sicurezza di un sistema informatico si effettua la cosiddetta
    \textbf{analisi del rischio}. L’analisi del rischio consiste nell’identificazione dei beni da proteggere,
    valutandone le possibili minacce in termini di probabilità di occorrenza e relativo danno
    potenziale. In base alla stima del rischio si decide se, come e quali contromisure di sicurezza
    adottare, giustificando i costi per la messa in sicurezza.\\

    L’analisi del rischio comprende:

    \begin{itemize}
        \item \textbf{Analisi delle vulnerabilità}: descrive le vulnerabilità di un sistema informatico;
        \item \textbf{Analisi degli attacchi}: descrive le tipologie di attacchi plausibili per le vulnerabilità
        trovate e, per ogni attacco, le informazioni e le risorse necessarie ad eseguirlo;
        \item \textbf{Analisi degli impatti}: per ogni attacco, si stabilisce la perdita dell’azienda;
        \item \textbf{Analisi delle minacce}: per ogni attacco, si stabilisce chi ha interesse nell’attacco e chi
        dispone delle risorse necessarie ad eseguirlo;
        \item \textbf{Individuazione del rischio accettabile ed introduzione delle contromisure}: si
        determinano le contromisure da attuare ed il rischio residuo, talvolta trasferibile a terzi
        (come le assicurazioni).
    \end{itemize}

    \subsection{SSL}

        \emph{Il funzionamento di SSL e TLS non sono stati trattati durante le lezioni, per cui si è evitato di
        spiegare i vari processi.}\\

        La sicurezza nel Web è un argomento molto vasto e sicuramente mettere in sicurezza le
        connessioni è uno degli obiettivi più importanti oggigiorno. Per raggiungere questo obiettivo, è
        stato inventato un pacchetto per la sicurezza chiamato \textbf{SSL (Secure Socket Layer)}, un
        encryption system a doppia chiave pubblica e privata, usato nei server per garantire la privacy
        durante le trasmissioni su Internet.\\

        SSL si pone in una sorta di strato intermedio della pila OSI, tra il livello Applicazione e il livello
        Trasporto, permettendo di cifrare le informazioni prima dell’invio ai client e evitando la lettura
        da parte di potenziali terzi in grado di intercettare il traffico tra server e client.\\

        Ogni server deve avere una doppia chiave (pubblica e privata) e un certificato X.509 usato per
        firmare i dati, rilasciato da una \emph{Certificate Authority (CA)}, degli enti considerati \emph{trusted}.\\

        Per effettuare una connessione con SSL, si deve:

        \begin{enumerate}
            \item Negoziare l’algoritmo di crittografia (come ad esempio DES o IDEA)
            \item Il server si autentica al client (sempre) ed il client si autentica al server (opzionalmente);
        \end{enumerate}

        Quindi è possibile inviare dati criptati con la chiave concordata.\\

        Si noti che HTTP usato sopra SSL prende il nuovo nome di \textbf{HTTPS (Secure HTTP)}, anche se si
        tratta ancora del protocollo HTTP standard ed è normalmente disponibile su una porta diversa,
        la \textbf{443}.\\

        Nel 1996, Netscape Corp. (creatrice di SSL), sottopose il proprio protocollo all’IETF per la
        standardizzazione. Il risultato fu \textbf{TLS (Transport Layer Security)}, anche noto con il nome di
        \emph{SSL 3.1}. I cambiamenti operati all’SSL furono minimali e hanno reso TLS un po’ più robusto, ma
        sono bastati a far sì che SSL versione 3 e TLS non riescano a comunicare tra loro. Non è ancora
        chiaro se TLS soppianterà SSL nella pratica.

    \subsection{Firewall}

        Un \textbf{firewall} è un sistema che ha lo scopo di controllare il traffico tra due o più reti sulla base di
        un insieme di regole. Si definiscono \emph{trusted} la rete interna da proteggere (solitamente una LAN)
        e \emph{untrusted} le reti esterne (generalmente una o più WAN). Può essere implementato sia lato
        hardware e software tramite un host con più schede di rete, costituendo il solo punto di
        collegamento tra le reti coinvolte, sia unicamente lato software sui vari host della rete interna
        da proteggere.\\

        Ogni socket che viene aperto dall’interno verso l’esterno (e viceversa) viene verificato in base
        alle regole definite all’interno del software, che viene detto \emph{packet filter}. Esistono due criteri
        generali per l’applicazione delle singole regole:

        \begin{itemize}
            \item \emph{default-deny}: viene \textbf{permesso} solo quanto dichiarato esplicitamente, il resto è \textbf{bloccato};
            \item \emph{default-allow}: viene \textbf{vietato} solo quanto dichiarato esplicitamente, il resto è \textbf{permesso}.
        \end{itemize}

        I firewall normalmente utilizzano il criterio \emph{default-deny}, in quanto garantisce una maggiore
        sicurezza e precisione nella definizione delle regole.

        \subsubsection{Packet Filter}

            \emph{Il packet filter richiederebbe una spiegazione breve sul funzionamento del software iptables.}\\

            Il packet filter è un tool che analizza l’intestazione dei vari pacchetti in entrata ed uscita e decide
            il da farsi in base a determinati criteri. Gli elementi utilizzati nel filtraggio sono:

            \begin{itemize}
                \item \textbf{Header IP}: che si ricorda contiene informazioni come indirizzo IP del mittente e del
                destinatario o il protocollo usato;
                \item \textbf{Header TCP/UDP}: che si ricorda contiene informazioni come le porte o le flag utilizzate.
            \end{itemize}

            Per ogni pacchetto è possibile effettuare una delle seguenti operazioni:

            \begin{itemize}
                \item \textbf{ACCEPT}: il pacchetto viene accettato e lasciato proseguire;
                \item \textbf{DROP}: il pacchetto viene scartato, senza mandare al mittente alcun messaggio;
                \item \textbf{REJECT}: come DROP, ma il mittente riceverà un pacchetto ICMP con un messaggio;
                \item \textbf{CHAIN} e \textbf{RETURN}: utilizzati per definire una concatenazione di regole.
            \end{itemize}

            L’insieme delle regole per il packet filtering è memorizzato in una tabella detta \textbf{filter}, che
            contiene a sua volta 3 diverse liste di regole, dette \textbf{chains}, distinte a seconda del percorso di
            routing del pacchetto:

            \begin{itemize}
                \item \textbf{INPUT}: riguarda i pacchetti che hanno come destinazione il firewall;
                \item \textbf{OUTPUT}: riguarda i pacchetti emessi dal firewall;
                \item \textbf{FORWARD}: riguarda i pacchetti che transitano attraverso il firewall.
            \end{itemize}

            Il packet filter più utilizzato (e preinstallato) su OS Linux è \textbf{iptables}.