\section{Definizioni Geneerali}
    \subsection{Rete}
        Una \textbf{rete} è un insieme di dispositivi collegati in modo da permettere lo scambio di dati e la
        comunicazione tra più utenti. I dati vengono trasferiti sotto forma di \textbf{pacchetti} (a volte detti
        \textbf{PDU}, \textbf{Packet Data Unit}).\\
        
        Si distingue da un sistema distribuito, che invece appare ai propri utenti come un singolo
        sistema coerente.

    \subsection{Internet}
        \textbf{Internet (o Internetwork)} è l’interconnessione di reti di varia natura che consente lo scambio
        di informazione rappresentata in forma digitale, cioè come sequenza di cifre binarie (bit).\\

        Si può dire che un \textbf{computer è su Internet} se esegue la pila di protocolli TCP/IP, ha un indirizzo
        IP, e può spedire pacchetti IP a tutti gli altri computer su Internet (questi argomenti saranno
        trattati nei capitoli successivi).

    \subsection{Extranet}
        L’\textbf{Extranet} è una Intranet estesa ad alcuni soggetti non operanti nella stessa rete (p.e. clienti,
        fornitori, consulenti).

    \subsection{ISP}
        Un \textbf{ISP (Internet Service Provider)} indica un’organizzazione o un’infrastruttura che mette a
        disposizione dei servizi inerenti a Internet per degli utenti, come la posta elettronica o l’accesso
        al World Wide Web.

    \subsection{Protocollo}
        Informalmente, un \textbf{protocollo} è un accordo, tra le parti che comunicano, sul modo in cui deve
        procedere la comunicazione. Violare il protocollo rende la comunicazione più difficile, se non
        del tutto impossibile.\\

        Formalmente, un \textbf{protocollo} è un \underline{insieme di regole} che definisce l’interazione tra sistemi.

    \subsection{Modello clinet/server}
        Il \textbf{modello client/server} è un modello in cui sono presenti due entità (client e server), nel
        quale il client si connette al server mediante una rete, per la fruizione di un certo servizio (come
        la condivisione dati).

    \subsection{Gateway}
        Un gateway è un dispositivo di rete (generalmente un router) che collega due reti eterogenee.
        Il suo scopo principale è quello di veicolare i pacchetti di rete all’esterno di una rete locale.

    \subsection{LAN}
        Le \textbf{LAN (Local Area Network)} sono reti private installate all’interno di un singolo edificio o
        campus, con dimensione \underline{fino a qualche Km}, ed hanno come scopo la condivisione di risorse
        (come stampanti) e lo scambio di informazioni.

    \subsection{MAN}
        Una \textbf{MAN (Metropolitan Area Network)} è una rete che copre un’intera città e collega più LAN
        geograficamente vicine. Di solito si tratta di singole filiali di un’azienda, che vengono connesse
        ad una MAN attraverso l’affitto di linee dedicate.

    \subsection{WAN}
        Una \textbf{WAN (Wide Area Network)} è una rete che copre un’area geograficamente estesa, spesso
        una nazione o un continente. Il numero di reti locali o singoli computer che si possono
        connettere ad una singola WAN è teoricamente illimitato.

    \subsection{Servizio connection oriented}
        Un servizio \textbf{connection oriented (orientato alla connessione)} è un servizio di rete in cui
        l’utente che lo vuole usare deve \underline{stabilire una connessione} (mediante la creazione di un circuito,
        che sia fisico o virtuale), \underline{usarla} e quindi \underline{rilasciarla}. Nella maggior parte dei casi, l’ordine dei bit
        inviati è conservato e arrivano nella sequenza con cui sono stati trasmessi, che rende questa
        categoria di servizi \textbf{affidabile}.\\

        Un’analogia utile per capire, è quella tra il servizio connection oriented e il \textbf{sistema telefonico}.
        Per parlare con qualcuno si deve prendere il telefono, comporre il numero, parlare e poi
        riagganciare.

    \subsection{Servizio connectionless}
        Un servizio \textbf{connectionless (senza connessione)} si contrappone al servizio con connessione
        e non viene quindi stabilita una connessione. I dati vengono instradati in maniera indipendente
        l’uno dall’altro, senza verificare che il destinatario sia raggiungibile e senza controllare che i
        dati arrivino nell’ordine desiderato, che rende questa categoria di servizi \textbf{inaffidabile}.\\

        Un’analogia utile per capire, è quella tra il servizio connectionless e la \textbf{posta}. Ogni messaggio
        (lettera) trasporta l’indirizzo completo del destinatario ed è instradato attraverso il sistema
        postale in modo indipendente dagli altri. Normalmente, quando si mandano due messaggi alla
        stessa destinazione, il primo inviato è anche il primo ad arrivare; ma è possibile che incontri un
        ritardo, e quindi arrivi dopo il secondo.

    \subsection{Servizio best effort}
        Un servizio \textbf{best effort} (\textbf{massimo impegno}, interpretato “come va, va”) è un servizio
        inaffidabile che non offre alcuna garanzia di consegna dei pacchetti (alcuni di essi possono
        perdersi e avere bisogno di essere ritrasmessi, o andare fuori sequenza) né di controllo di
        errore, di flusso e di congestione.

    \subsection{Quality of Service}
        Il termine \textbf{Quality of Service} (abbreviato \textbf{QoS}) è utilizzato per indicare i parametri usati per
        caratterizzare la qualità del servizio offerto da una rete (ad esempio perdita di pacchetti,
        ritardo), o gli strumenti o tecniche per ottenere una qualità di servizio desiderata. Si
        contrappone al termine best effort.

    \subsection{Unicsting, Multicasting e Broadcasting}
        Unicasting, Multicasting e Broadcasting sono delle metodologie per la trasmissione dei dati.
        La distinzione avviene in base al numero dei ricevitori:

        \begin{itemize}
            \item \textbf{Unicasting}: trasmissione di dati tra un trasmettitore e un ricevitore (1-1);
            \item \textbf{Multicasting}: trasmissione di dati tra un trasmettitore ed un sottoinsieme di tutte le macchine della rete (1-n);
            \item \textbf{Broadcasting}: trasmissione di dati tra un trasmettitore e tutte le macchine della rete (1-tutti).
        \end{itemize}

    \subsection{RFC}
        Gli \textbf{RFC (Request For Comment)}, sono una serie di rapporti tecnici numerati
        cronologicamente che riportano informazioni o specifiche riguardanti nuove ricerche,
        innovazioni e metodologie di Internet. Sono redatti da un organismo internazionale chiamato
        \textbf{IETF (Internet Engineering Task Force)}, e sono tutti consultabili su \url{www.ietf.org/rfc/}.
