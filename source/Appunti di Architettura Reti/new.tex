\section{News}

    Molto tempo prima di Internet esisteva il sistema di news \textbf{USENET}, un sistema distribuito per
    il dialogo ed il confronto tra utenti. Consiste in circa 30.000 \textbf{newsgroup}, gruppi di discussione
    gerarchici in cui si inviano e leggono articoli correlati. Una newsgroup è indicata nella forma:
    \textbf{radice.gruppo.sottogruppo}, con un numero di sottogruppi indefinito. Un utente vi si può
    iscrivere, in modo da ricevere la lista dei messaggi del gruppo, e a differenza delle mail, le news
    vengono scaricate sul client solo nel momento in cui l’utente vi accede.\\

    Quando un utente invia un messaggio ad un newsgroup, si dice che effettua un \textbf{post}. Nel caso
    in cui i messaggi vengono inviati a diversi newsgroup, si parla di \textbf{crosspost}. I messaggi nei
    gruppi sono però \textbf{moderati}, ovvero passano per un intermediario, detto \textbf{moderatore}, che
    determina la pubblicazione o il rifiuto di un articolo. Questa scelta viene anche basata sulla
    \textbf{netiquette}: delle regole informali che disciplinano il buon comportamento di un utente.\\

    In alcuni casi, come la creazione di un particolare newsgroup, è possibile indire una \textbf{CfV (Call
    for Votes)}, dove gli utenti possono esprimere le loro preferenze. Generalmente, le CfV vengono
    effettuate sotto la radice \textbf{news}, mentre sotto la radice \textbf{alt} c’è più libertà, ed è in genere dove si
    può trovare di tutto, compreso materiale al limite della legalità.\\

    Essendo un sistema distribuito, è necessario scambiare i messaggi tra i vari server dislocati. Il
    meccanismo per assolvere questo compito è detto \textbf{newsfeed}, dove il server che riceve il
    messaggio, lo inoltra ai server vicini, i quali a loro volta effettueranno le stesse operazioni fino
    a raggiungere tutti. I messaggi vengono archiviati nei server fino ad una data di scadenza
    prefissata e variabile da gruppo a gruppo, momento nel quale vengono fisicamente cancellati.

    \subsection{NNTP}

        \textbf{NNTP (Network News Transfer Protocol)} è il protocollo client/server che regola lo scambio
        dei messaggi tra i server nel meccanismo di newsfeed. Nello specifico, definisce distribuzione,
        interrogazione, accesso e invio di news. Utilizza TCP e dialoga sulla porta 119.
        Quando c’è un nuovo messaggio, un server informa gli altri tramite il comando \textbf{IHAVE}.