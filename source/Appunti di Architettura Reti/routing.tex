\section{Routing}
    Il \textbf{routing} riguarda il livello 3 del modello di riferimento ISO-OSI ed è l’azione d’\textbf{instradare i
    pacchetti dal computer sorgente al computer di destinazione}. Nella maggior parte delle
    sottoreti i pacchetti compiono \textbf{salti multipli} per raggiungere la meta; l’unica eccezione degna
    di nota riguarda le reti broadcast, ma anche in questo caso se sorgente e destinazione non si
    trovano sulla stessa rete, il routing è un problema. Consiste in due attività principali:\\
    
    \begin{itemize}
        \item Determinare il percorso migliore;
        \item Trasportare pacchetti sulla rete.
    \end{itemize}

    Per riportare eventuali \textbf{malfunzionamenti} nel routing, viene usato \textbf{ICMP}. In particolare, il
    messaggio \textbf{redirection} indica la necessità di reinstradare i pacchetti in modo migliore, cioè
    segnala che un router è stato attraversato inutilmente.

        \subsection{Algoritmi di routing}
        L’\textbf{algoritmo di routing} è quella parte del software del livello network, che si preoccupa di
        scegliere lungo quale linea di uscita vanno instradati i pacchetti in arrivo.
        Ad un algoritmo di routing devono appartenere le seguenti \textbf{proprietà}:
        
        \begin{itemize}
            \item \textbf{Semplicità}: consumo di meno risorse possibile;
            \item \textbf{Robustezza e stabilità}: comportamento corretto in condizioni inusuali (guasti
            hardware/software) e mai viste prima, reagendo in modo opportuno a congestioni;
            \item \textbf{Imparzialità}: la strada scelta non deve intenzionalmente avvantaggiare alcun nodo;
            \item \textbf{Ottimalità}: riduzione al minimo possibile del ritardo medio dei pacchetti,
            massimizzazione delle capacità di carico totale della rete (anche se molto spesso questi
            due obiettivi sono in conflitto ed occorre scendere a compromessi).
        \end{itemize}
        
        A seconda della tipologia di algoritmo scelto, il routing si distingue in \textbf{tre classi principali}:

        \begin{itemize}
            \item \textbf{Minimale}: le decisioni sono tutte definite al momento della configurazione
            dell’interfaccia di rete;
            \item \textbf{Statico}: utilizza algoritmi \textbf{non adattivi}, ovvero che non basano le loro decisioni su
            misure o stime del traffico e della topologia corrente. Il percorso utilizzato per collegare
            due nodi è calcolato in anticipo, in modalità offline, ed è scaricato nei router all’avvio
            della rete;
            \item \textbf{Dinamico}: utilizza algoritmi \textbf{adattivi}, che cambiano le loro decisioni secondo le
            modifiche apportate alla topologia e di solito anche al traffico.
        \end{itemize}

        Uno stato importante per gli algoritmi di routing è lo stato di \textbf{convergenza}: diremo che un set
        di router sono in uno stato di convergenza, se le informazioni topologiche riguardo la
        internetwork dove operano è la medesima. Analizzare quanto velocemente un algoritmo di
        routing raggiunge la convergenza è molto utilizzato come metro di paragone.

            \subsubsection{Routing table}
            Una \textbf{routing table} (o \textbf{tabella di routing}) è un database memorizzato in un router o in un host
            che elenca i vari \textbf{next hop} (\textbf{prossimo salto}, il prossimo nodo da attraversare) a cui può andare
            un pacchetto. Quindi nella tabella di routing, ogni riga corrisponde ad una possibile strada, i cui
            elementi essenziali sono:
            
            \begin{itemize}
                \item \textbf{ID di rete}: l’indirizzo dell’host o sottorete di destinazione;
                \item \textbf{Next hop}: indirizzo dell’eventuale prossimo gateway da attraversare;
                \item \textbf{Costo/metrica}: il costo o metrica del percorso. Questa unità di misura viene detta
                \textbf{metric} e non è univocamente definibile: può assumere infatti diversi significati:
                    \begin{itemize}
                        \item \textbf{Path Length (o hop-count)}: numero di nodi da attraversare;
                        \item \textbf{Reliability}: l’affidabilità di una rete misurata ad intervalli costanti;
                        \item \textbf{Delay}: il tempo necessario ad ogni pacchetto per raggiungere la destinazione;
                        \item \textbf{Bandwidth}: l’ampiezza di banda;
                        \item \textbf{Load}: traffico misurato ad intervalli regolari;
                        \item \textbf{Communication Cost}: numero intero arbitrario che indica quanto un percorso
                        sia conveniente: valori più bassi indicano un percorso migliore.
                    \end{itemize}
            \end{itemize}

            Le routing table possono ricevere informazioni da due sorgenti:

            \begin{itemize}
                \item Dal \textbf{file di configurazione} creato dall’amministratore di rete, salvato sul disco del
                dispositivo e interpretato in fase di inizializzazione dell’hardware;
                \item Da \textbf{protocolli dinamici di aggiornamento} (protocolli di routing).
            \end{itemize}

            Alla ricezione di un pacchetto, ogni nodo della rete controlla la classe dell’indirizzo IP o
            sottorete di destinazione:

            \begin{itemize}
                \item se l’indirizzo \textbf{è locale}, allora applica la subnet mask alla destinazione e invia
                direttamente all’host destinatario;
                \item se invece \textbf{non è locale}, allora cerca nella routing table la rete di destinazione e instrada
                il datagram verso il gateway corrispondente.
            \end{itemize}

            In ambiente \textbf{Unix}, il comando per visualizzare la tabella di routing è: \textbf{`netstat -r`}.
            Opzionalmente, è possibile usare la specifica -n per ottenere gli indirizzi di destinazione in
            forma numerica. Per ogni risultato può comparire un campo flag, i cui significati sono i seguenti:

            \begin{itemize}
                \item \textbf{U}: indica che l’interfaccia di rete è attiva (infatti U sta per UP);
                \item \textbf{G}: indica un’uscita verso un’altra rete tramite gateway;
                \item \textbf{H}: indica che la destinazione è l’indirizzo completo di un host;
                \item \textbf{D}: indica una route aggiunta da un ICMP redirect.
            \end{itemize}
            
            \subsubsection{Routing in una intranetwork}
            Il routing attraverso una internetwork è simile al routing eseguito in una singola sottorete, ma
            con alcune complicazioni in più. Il routing viene infatti effettuato a due livelli:

            \begin{itemize}
                \item dentro ogni rete viene utilizzato un protocollo di gateway interno
                (\textbf{IGP}: Interior Gateway Protocol);
                \item tra le reti è adottato un protocollo di gateway esterno (\textbf{EGP}: Exterior Gateway Protocol).
            \end{itemize}

            Esiste inoltre un’altra famiglia di protocolli di routing, che non rientra né in quella IGP né in
            quella EGP, ed è la \textbf{CIDR (Classless Inter-Domain Routing)}.\\

            Poiché ciascuna rete è indipendente, ognuna può usare algoritmi differenti e a forte di questa
            loro indipendenza, un gruppo di router e reti sotto il controllo di una singola autorità
            amministrativa ben definita, viene chiamato \textbf{Autonomous System (AS)}. Ogni AS ha un numero
            identificativo univoco, detto \textbf{ASN (Autonomous System Number)}, utile ai fini del routing. Gli
            ASN sono assegnati da ICANN ai Regional Internet Registries (RIRs).\\
            
            Gli AS possono essere suddivisi in:

            \begin{itemize}
                \item \textbf{multihomed autonomous system}: è un AS che mantiene connessioni con più di un AS;
                \item \textbf{stub autonomous system}: si riferisce ad un AS connesso solamente con un altro AS;
                \item \textbf{transit autonomous system}: è un AS che fornisce attraverso di sé connessioni con altre
                reti
            \end{itemize}

        \subsection{Classificazione di algoritmi adattivi}
        Le moderne reti di computer generalmente utilizzano algoritmi di routing dinamici al posto di
        quelli statici, poiché gli algoritmi statici non tengono conto del carico istantaneo della rete.
        I due algoritmi dinamici più popolari sono il routing basato sul vettore delle distanze e il routing
        basato sullo stato dei collegamenti.

        \subsection{Algoritmi di tipo distance-vector (vettore-distanza)}
        Gli algoritmi di routing basati sul \textbf{vettore delle distanze} (distance vector routing) operano
        facendo in modo che ogni router conservi una tabella (ossia un \underline{vettore}), che definisce la
        migliore \underline{distanza} conosciuta per ogni destinazione e la linea che conduce a tale destinazione.
        Queste tabelle sono aggiornate scambiando informazioni con i router vicini, e gli aggiornamenti
        sono in forma di coppie del tipo \textbf{(rete di destinazione, distanza)}, dove la metrica usata per la
        distanza potrebbe essere il numero di salti, il ritardo espresso in millisecondi, il numero totale
        di pacchetti accodati lungo il percorso o qualcosa di simile.

        \subsection{Algoritmi di tipo link-state (stato dei collegamenti)}
        Il routing basato sul vettore delle distanze è stato utilizzato in ARPANET fino al 1979, quando
        fu sostituito dal routing basato sullo \textbf{stato dei collegamenti}, ancora largamente usato in
        diverse varianti. L’idea alla base di questo algoritmo è semplice e può essere riassunta in cinque
        punti. Ogni router deve:

        \begin{enumerate}
            \item scoprire i propri vicini e i relativi indirizzi di rete;
            \item misurare il ritardo (delay) o il costo di ogni vicino;
            \item costruire un pacchetto detto link-state che contiene tutte le informazioni raccolte;
            \item inviare questo pacchetto a tutti gli altri router;
            \item elaborare il percorso più breve verso tutti gli altri router.
        \end{enumerate}

        In pratica, si misurano sperimentalmente la topologia completa e tutti i ritardi per poi
        distribuirli a ogni router; successivamente viene eseguito l’algoritmo di Dijkstra per trovare il
        percorso più breve associato a ogni altro router.\\
        
        Questo tipo di routing permette di effettuare calcolo distribuito (ogni router lavora), evita i
        problemi che si hanno quando un router trasmette informazioni errate e può gestire reti
        composte da un gran numero di nodi

        \subsection{OSPF}
        Sviluppato nel 1988 da un sottogruppo di IEFT (Internet Engineering Task Force), \textbf{OSPF (Open
        Shortest Path First)} è oggi supportato dalla maggior parte dei produttori di router ed è
        diventato il principale protocollo di \textbf{routing per gateway interni}.\\
        
        Si tratta di un protocollo aperto (\textbf{Open}, senza vincoli di brevetto) che soddisfa alcuni requisiti:
        
        \begin{itemize}
            \item supporta \textbf{diverse metriche di distanza} (come la distanza fisica, il ritardo...);
            \item è un algoritmo \textbf{dinamico}, in grado di modificarsi rapidamente e automaticamente in
            base alla topologia;
            \item è stato realizzato come supporto al \textbf{routing basato sul tipo di servizio} (ottenibile dal
            campo Type of Service di IP), ma a causa del suo scarso utilizzo (in sua assenza gli utenti
            avevano già trovato altre soluzioni), si è deciso di eliminarlo in quanto superfluo;
            \item esegue il \textbf{bilanciamento del carico}, ossia suddividere il carico su più linee: la maggior
            parte dei protocolli precedenti inviava tutti i pacchetti lungo il percorso migliore ed il
            percorso successivo al migliore non era mai utilizzato;
            \item supporta i \textbf{sistemi gerarchici (aree)}, in quanto prima nessun router poteva (né
            doveva) conoscere l’intera topologia di Internet, a causa della sua dimensione elevata e
            crescente;
            \item supporta l’\textbf{autenticazione dei messaggi}, migliorando la sicurezza ed impedendo agli
            utenti di imbrogliare i router inviando loro false informazioni di routing.
        \end{itemize}

        OSPF opera rappresentando la \textbf{rete reale come un grafo} e poi elabora il \textbf{percorso più breve}
        da ogni router a ogni altro router, in base al costo degli archi (distanza, ritardo e così via).\\
        
        Siccome molti AS di Internet sono a loro volta grandi e difficili da gestire, OSPF permette di
        dividere tali AS in più \textbf{aree}, dove ogni area è una rete o un insieme di reti contigue. Ogni router
        conserva il \textbf{topological database} della sua area, ovvero una lista dei router adiacenti,
        indicando quelli designati a svolgere il ruolo di “rappresentanti dell’area”, cioè \textbf{DR (Designated
        Routers)} e \textbf{BDR (Backup Designated Routers)}. Ogni AS ha un’area dorsale chiamata
        \textbf{backbone area} o \textbf{area 0} e tutte le aree sono collegate alla dorsale, il che permette di passare
        da un’area a qualunque altra dello stesso AS attraverso la backbone.
        I router della stessa area hanno lo stesso topological database, e la divisione in aree permette
        di ridurre il traffico di routing. L’OSPF backbone distribuisce le informazioni di routing tra le
        aree e ne può apprendere ulteriori mediante istruzioni di configurazione o tramite EGP o IGP.\\

        La divisione in aree porta a 2 tipi diversi di routing (\textbf{intra-area} ed \textbf{inter-area}) e a 4 diversi tipi
        di router:
        
        \begin{itemize}
            \item \textbf{IR (Internal Router)}: router completamente interno ad un’area;
            \item \textbf{ABR (Area Border Router)}: collegano due o più aree, mantengono il topological
            database di ogni area che collegano;
            \item \textbf{BR (Backbone Router)}: router dell’area di backbone;
            \item \textbf{ASBR (AS Boundary Router)}: comunicano con i router che si trovano in altri AS.
        \end{itemize}
        
        OSPF è un protocollo \textbf{intra-AS (IGP)} (anche se può funzionare tra diversi AS) e \textbf{link-state},
        poiché invia \textbf{link-state advertisements (LSA)} a tutti i router della stessa area per diffondere
        informazioni sulle interfacce attive, sulle metric e altre informazioni.\\
        
        I \textbf{pacchetti di tipo LSA} di OSPF si dividono in \textbf{5 tipi}:
        
        \begin{itemize}
            \item \textbf{Hello}: usato per scoprire chi sono i vicini;
            \item \textbf{Database Description}: fornisce i propri criteri per la selezione del costo del link;
            \item \textbf{Link State Request}: richiesta di informazioni di stato ai router vicini;
            \item \textbf{Link State Update}: comunica i costi utilizzati nel database della topologia;
            \item \textbf{Link State Acknowledgement}: conferma un Link State Update.
        \end{itemize}
        
        \subsection{RIP}
        \textbf{RIP (Routing Information Protocol)} è un popolare algoritmo di routing che utilizza UDP come
        protocollo di trasporto (porta 520) e si basa sull’algoritmo vettore-distanza utilizzando il
        numero di hop come metric.\\
        
        RIP presenta un \textbf{limite di 15 hop}: reti più lontane di 15 hop (indicate con 16 hop) sono
        considerate irraggiungibili e considerate aventi metrica infinita. Nonostante ciò, non riesce ad
        evitare i routing loop. Questi sono alcuni dei motivi che hanno portato all’utilizzo ridotto di RIP,
        oltre al fatto che converge più lentamente rispetto ad altri algoritmi come OSPF e non supporta
        subnet variabili. RIP è più leggero di OSPF (richiede meno risorse) ed è disponibile per default
        sui sistemi Unix/Linux (daemon \textbf{`routed`}).\\
        
        Il funzionamento di RIP è semplice: i router che lo implementano, inviano tutta la routing table
        o parte di essa ai router vicini ad intervalli di tempo regolari, dopodiché ogni router si tiene
        solo le informazioni relative alle best routes. Il RIP ha due forme a seconda dell’utente:\\
        
            \begin{itemize}
                \item Forma \textbf{passiva}: usata dagli host, riceve messaggi ma non ne invia;
                \item Forma \textbf{attiva}: usata dai router, riceve ed invia in broadcast messaggi.
            \end{itemize}
        
        Questi messaggi prendono il nome di \textbf{routing updates} e consistono in una porzione della
        routing table in cui si è trovato un percorso migliore. Gli aggiornamenti vengono inviati in due
        occasioni:
        
            \begin{itemize}
                \item Ad ogni tick del routing-update timer, cioè \textbf{a cadenza regolare} (di default 30 secondi);
                \item Quando \textbf{cambia la topologia} della rete dei router confinanti.
            \end{itemize}
            
        Un altro timer utilizzato è il \textbf{route timeout}: se una route non viene aggiornata nella tabella
        entro un tempo limite, viene segnata come non valida e successivamente rimossa allo scadere
        del \textbf{route-flush} timer.\\

        Per evitare di annunciare false informazioni di routing, RIP implementa alcune tecniche:

            \begin{itemize}
                \item \textbf{Split horizon}: il router non propaga informazioni su una route al router che ha generato
                tale aggiornamento;
                \item \textbf{Hold down}: costringe un router a ignorare aggiornamenti inerenti a una rete per un
                certo tempo (60 sec. in genere), dopo aver ricevuto un messaggio di rete irraggiungibile;
                \item \textbf{Poison reverse}: quando un collegamento scompare, il router che effettuava l’annuncio
                continua ad annunciarlo ancora per un certo tempo, assegnandoli distanza infinita
            \end{itemize}

        \subsection{BGP}
        Dentro un singolo AS, il protocollo di routing raccomandato è OSPF (sebbene non sia l’unico
        utilizzato). Tra AS si utilizza un protocollo diverso, chiamato \textbf{BGP (Border Gateway Protocol)},
        perché gli obiettivi sono differenti: un protocollo per gateway interni non deve far altro che
        spostare i pacchetti nel modo più efficiente possibile dalla sorgente alla destinazione; non deve
        preoccuparsi della \textbf{politica}, ovvero vincoli che si possono imporre. Alcuni esempi di vincoli:
        
            \begin{itemize}
                \item Nessun traffico di passaggio attraverso certi AS;
                \item Mai mettere l’Iraq su un percorso che inizia dal Pentagono;
                \item Passare per l’Albania solo se non esistono alternative per la destinazione.
            \end{itemize}

        In genere i criteri sono configurati manualmente in ogni router BGP (o inseriti usando qualche
        tipo di script), e non fanno parte del protocollo stesso.\\

        Dato l’interesse di BGP verso il traffico di passaggio, le reti sono raggruppate in 3 categorie:

            \begin{itemize}
                \item \textbf{Stub networks (reti terminali)}: sono reti che non possono essere utilizzate per il
                traffico di passaggio perché non ci sono altre reti a cui si collega;
                \item \textbf{Multiconnected networks (reti multicollegate)}: reti utilizzate per il traffico di
                passaggio, se non rifiutano di farlo;
                \item \textbf{Transit networks (reti di transito)}: simili alle reti backbone nell’OSPF, sono reti
                pronte a gestire pacchetti di terze parti, di solito con alcune restrizioni o a pagamento.
            \end{itemize}


        BGP effettua \textbf{interdomain routing (EGP)} basandosi su un algoritmo \textbf{vettore-distanza} (un po’
        diverso da quello applicato ad altri protocolli) e utilizzando connessioni TCP (porta 179), che
        rende le comunicazioni affidabili e nasconde tutti i dettagli della rete attraversata.\\
        

        Infatti, l’algoritmo \textbf{vettore-distanza} utilizzato in BGP è stato \textbf{modificato} in modo da tenere
        traccia anche del percorso utilizzato, anziché solo il costo di ogni destinazione. Inoltre, anziché
        comunicare periodicamente a ogni vicino il costo stimato associato a ogni possibile
        destinazione, ogni router BGP comunica l’esatto percorso in uso e basta. Ogni router BGP
        contiene un modulo che esamina i percorsi diretti a una destinazione e assegna loro un
        punteggio (distanza associata a quella destinazione) e nel caso un percorso violi un criterio di
        vincolo riceve punteggio $\infty$. Tuttavia la funzione che valuta il punteggio dei percorsi non fa
        parte del protocollo BGP e può essere scelta dagli amministratori del sistema.\\

        I peer che eseguono il protocollo BGP eseguono 3 funzioni di base:

            \begin{itemize}
                \item \textbf{Autenticazione reciproca}, per stabilire la correttezza delle operazioni e vedere se è
                possibile comunicare;
                \item \textbf{Scambio di informazioni}, come le reti raggiungibili (e non) o dati del salto successivo;
                \item \textbf{Funzioni di verifica}, per controllare la correttezza dei peer e della connessione di rete.
            \end{itemize}


        Per definire queste 3 funzioni, BGP definisce un insieme di 5 messaggi:

            \begin{itemize}
                \item \textbf{Open}: primo messaggio inviato da ciascuna parte dopo l’attivazione della sessione TCP,
                serve ad aprire una sessione tra peer. Deve essere confermato da un messaggio
                keepalive prima che possa aver luogo la comunicazione;
                \item \textbf{Update}: aggiornamento del routing, consente al router di avere una mappa della rete ed
                eventualmente cancellare reti irraggiungibili;
                \item \textbf{Notification}: inviato in condizione di errore, chiude una sessione attiva e avvisa i router
                vicini del motivo della chiusura;
                \item \textbf{Keepalive}: messaggi di servizio inviati regolarmente per mantenere la connessione;
                \item \textbf{Refresh}: chiede il reinvio delle informazioni di routing per una rete da parte di un peer.
            \end{itemize}

                
        Il protocollo BGP può essere utilizzato in 3 differenti scenari:

            \begin{itemize}
                \item \textbf{Inter-Autonomous System Routing (eBGP)}: routing tra 2 o più router BGP di AS
                diversi;
                \item \textbf{Intra-Autonomous System Routing (iBGP)}: routing tra 2 o più router BGP di uno
                stesso AS;
                \item \textbf{Pass-through Autonomous System Routing}: routing tra 2 o più router BGP che
                scambiano traffico attraverso un AS che non esegue BGP.
            \end{itemize}