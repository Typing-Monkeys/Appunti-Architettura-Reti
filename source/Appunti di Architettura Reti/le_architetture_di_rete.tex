\section{Le Architetture di Rete}
    Esistono due importanti architetture di rete: il modello di riferimento \textbf{OSI} e il modello \textbf{TCP/IP}.
    Anche se i protocolli associati al modello OSI ormai sono in disuso, il modello in sé ha valore
    generale ed è ancora valido, e le caratteristiche discusse per ogni livello sono ancora molto
    importanti. Il modello TCP/IP ha caratteristiche opposte: il modello in sé è poco utilizzabile, ma
    i protocolli sono largamente impiegati; per questo motivo vanno esaminati entrambi in
    dettaglio. A volte s’impara più dai fallimenti che dai successi.
    
    \begin{tabular}{ll}
        \textit{OSI} & \textit{TCP/IP} \\
        \begin{tabular}{|c|c|}
            \hline
            \textbf{Livello 7} & Applicazione \\
            \hline
            \textbf{Livello 6} & Presentazione \\
            \hline
            \textbf{Livello 5} & Sessione \\
            \hline
            \textbf{Livello 4} & Trasporto \\
            \hline
            \textbf{Livello 3} & Rete \\
            \hline
            \textbf{Livello 2} & DataLink \\
            \hline
            \textbf{Livello 1} & Fisico \\
            \hline
        \end{tabular} 
        &
        \begin{tabular}{|c|}
            \hline
            Applicazione \\
            \hline
            Presentazione \\
            \hline
            (non presente) \\
            \hline
            (non presente) \\
            \hline
            Trasporto \\
            \hline
            Internet \\
            \hline
            Host-to-network \\
            \hline
        \end{tabular}
    \end{tabular}

        \subsection{Il modello di riferimento ISO/OSI}
        L’ISO (International Standards Organization) è un organo consulente dell’ONU, che promuove
        lo sviluppo di standardizzazioni nel mondo. È il creatore dell’\textbf{OSI (Open Systems
        Interconnections)}, così chiamato perché riguarda la connessione di sistemi aperti, cioè sistemi
        che sono “aperti” verso la comunicazione con altri. È uno standard \textbf{\textit{de jure}} ("per legge", ovvero
        formale, adottato da qualche organismo di standardizzazione autorizzato, in questo caso ISO).
        È caratterizzato da \textbf{7 livelli}: ogni livello sfrutta i servizi dei livelli inferiori. La comunicazione
        tra livelli adiacenti avviene tramite i \textbf{NAP (Neutral Access Points)}, mentre la comunicazione
        tra entità di livelli diversi avviene tramite il \textbf{SAP (Service Access Point)}. Le operazioni
        specifiche di un livello sono realizzate tramite un insieme di protocolli. Di seguito una breve
        descrizione di ogni livello:

        \begin{itemize}
            \item \textbf{Physical Layer}: si occupa della trasmissione di bit grezzi sul canale di comunicazione.
                È l’insieme di regole che specificano le connessioni elettriche e fisiche tra i dispositivi. Definisce
                la specifica dei cavi e del tipo di segnale elettrico associato ai vari pin.
                
            \item \textbf{Data Link Layer}: definisce l’accesso al mezzo specificato nel Physical Layer, il formato dei dati
                ed è responsabile dell’invio affidabile delle informazioni, ovvero gestisce tra le altre cose la
                frammentazione dei dati e le procedure di controllo di possibili errori del livello fisico.
                Appartengono a questo livello i protocolli Data Link (DLCP, BSC, HDLC...), sono inoltre presenti
                anche i sottolivelli LLC e MAC e hardware come Network Interface Card (NIC), Hub, Switch (di
                livello 2) e Bridge.
                
            \item \textbf{Network Layer}: si occupa della connessione tra due nodi della rete (detti nodo sorgente e nodo
                destinatario). Gestisce il routing e lo scambio di informazione tra i nodi. I servizi associati a
                questo livello sono legati al movimento dei dati nella rete, inclusi l’indirizzamento, il routing e
                le procedure di controllo dei flussi (vengono definite le raccomandazioni X.25 e X.75).
                A questo livello appartiene il protocollo IP.
                
            \item \textbf{Transport Layer}: è il garante del trasferimento delle informazioni. Analizza il traffico fra i nodi
                controllando gli errori, la sequenza e i fattori di affidabilità dello scambio.
                A questo livello appartengono i protocolli TCP e UDP. È il primo livello \textbf{end-to-end}.
                
            \item \textbf{Session Layer}: regolarizza l’inizio e la fine dei flussi dei dati fra i nodi. Si occupa
                dell’organizzazione del dialogo tra i programmi applicativi.
                
            \item \textbf{Presentation Layer}: si occupa di effettuare trasformazione sui dati compatibilmente con il
                dispositivo di ricezione. Esempi di trasformazione riguardano crittografia e compressione.

            \item \textbf{Application Layer}: è l’ultimo livello, comprende tutti i programmi applicativi che consentono
                l’uso della rete: fa da interfaccia tra rete e utente. Esempi di funzioni svolte da questo livello
                possono essere: terminale virtuale, trasferimento di file, posta elettronica, condivisione di
                risorse e accesso a database.
        \end{itemize}

        \subsection{Il modello di riferimento TCP/IP}
        Il \textbf{TCP/IP} è così chiamato per ricordare i suoi protocolli più importanti: il Transmission Control
        Protocol (TCP) e l'Internet Protocol (IP). È uno standard \textbf{\textit{de facto}} ("dalla realtà", ovvero
        stabilito senza piani formali). Di seguito una breve descrizione di ogni livello:
        \begin{itemize}
        
            \item \textbf{Host-to-network Layer}: “il grande vuoto”, nel modello di riferimento TCP/IP non viene
                specificato cosa accade in questo territorio, si limita a segnalare che l’host deve collegarsi alla
                rete usando qualche protocollo che gli permetta di spedire pacchetti IP. Questo protocollo però
                non è definito e varia da host a host.
                
            \item \textbf{Internet Layer}: il suo scopo è quello di consentire agli host di mandare pacchetti in qualsiasi
                rete, e farli viaggiare in modo indipendente l’uno dall’altro fino alla destinazione (che magari è
                su una rete diversa). È importante notare che viene definito il protocollo \textbf{IP}.
            
            \item \textbf{Transport Layer}: è progettato per consentire la comunicazione tra entità pari degli host
                sorgente e destinazione, come nel livello trasporto OSI. Sono definiti due protocolli di trasporto
                end-to-end: \textbf{TCP} e \textbf{UDP}.

            \item \textbf{Application Layer}: nel modello TCP/IP non ci sono sessione e presentazione, poiché si è notato
                che nella maggior parte dei casi sono inutili. Come nel modello OSI, il livello Applicazione
                contiene un gran numero di programmi che si interfacciano con l’utente.
        \end{itemize}

    \section{Internet Protocol(IP)}
    L’IP è la colla che tiene unito Internet. È il protocollo \textbf{connectionless e best effort} del livello 3
    (Rete) e permette di interconnettere reti eterogenee: per questo è implementato sopra ai
    protocolli di livello Data Link, svolgendo la funzione di routing (ovvero scegliere il percorso che
    dovranno seguire i dati, spiegato in seguito).
        
        \subsection{Datagram IP}
        Il pacchetto utilizzato dall’IP è detto \textbf{datagram IP} (per analogia con il servizio telegrafico,
        telegram service) ed è costituita da una parte di intestazione (detta \textbf{header}) e da una parte di
        testo (detta \textbf{data}). La versione attualmente utilizzata è la v4, anche se al momento è in corso
        una transizione da IPv4 a IPv6, che ha avuto inizio anni fa e il cui completamento richiederà
        ancora parecchio tempo (o addirittura mai). Di seguito viene analizzato l’header dell’IPv4.
    
            \subsubsection{IPv4}
            L’intestazione ha una parte fissa di 20 byte e una parte opzionale di lunghezza variabile ed è
            trasmessa in ordine \textit{big endian} (da sinistra a destra). Sulle macchine \textit{little endian} è necessario
            eseguire una conversione software sia in ricezione che in trasmissione.\\
           
            \begin{tikzpicture}
                \coordinate (A) at (-7,0);
                \coordinate (B) at ( 7.8,0);
                \draw[<->] (A) -- (B) node[midway,fill=white] {32 bit};
            \end{tikzpicture}
            
            \begin{tabular}{|l|l|l|l|l|l|l|l|l|l|l|l|l|l|l|l|l|l|l|l|l|l|l|l|l|l|l|l|l|l|l|l|}
                \hline
                & & & & & & & & & & & & & & & & & & & & & & & & & & & & & & & \\
                \hline
                \multicolumn{32}{l}{} \\
                \hline
                \multicolumn{4}{|c|}{Version} & \multicolumn{4}{|c|}{IHL} & \multicolumn{6}{|c|}{Type of Service} & & & \multicolumn{16}{|l|}{Total Lenght}\\
                \hline
                \multicolumn{16}{|c|}{Identification} & & DF & MF & \multicolumn{13}{|c|}{Fragment Offset} \\
                \hline
                \multicolumn{8}{|c|}{Time to Live} & \multicolumn{8}{|c|}{Protocol} & \multicolumn{16}{|c|}{Header Checksum}\\
                \hline
                \multicolumn{32}{|c|}{Source address}\\
                \hline
                \multicolumn{32}{|c|}{Destination address}\\
                \hline
                \multicolumn{32}{|c|}{Opzioni (0 o più word)}\\
                \hline
            \end{tabular}

            \begin{itemize}
                \item \textbf{Version}: campo di 4 bit che indica la versione IP del datagram (attualmente IPv4);

                \item \textbf{IHL (Internet Header Length)}: campo di 4 bit che indica la lunghezza dell’header
                    espressa in parole da 32 bit. Necessario poiché, come anticipato, la lunghezza
                    dell’intestazione non è costante;

                \item \textbf{Type of Service}: campo di 8 bit che indica come deve essere gestito il datagram;

                \item \textbf{Total Length}: campo lungo 16 bit che identifica la lunghezza totale del datagram;

                \item \textbf{Identification, flag, offset}: controllano frammentazione e riassemblaggio del datagram;

                \item \textbf{TTL (Time To Live)}: indica la durata in secondi concessa al datagram per restare in rete,
                    evitando che un datagram giri a vuoto per sempre, evento che potrebbe accadere in caso
                    di danneggiamento delle tabelle di routing;

                \item \textbf{Protocol}: è un codice che identifica il protocollo utilizzato nel campo dati
                    (intuitivamente, può essere associato all’estensione nel nominativo di un file);

                \item \textbf{Header Checksum}: garantisce il controllo dell’integrità dell’header;

                \item \textbf{Source address, Destination address}: rispettivamente indirizzo sorgente e indirizzo di
                    destinazione, indicano il numero di rete e il numero di host, espressi tramite indirizzi IP;

                \item \textbf{Options (opzioni)}: via di fuga per dare alle versioni successive del protocollo la
                    possibilità d’includere informazioni non presenti nel progetto originale
            \end{itemize}
            
            \subsubsection*{Indirizzo IPv4}
            Un indirizzo IP (in entrambe le sue versioni) permette di identificare univocamente \textbf{la
            connessione di un host alla rete}. L’indirizzo IPv4 è espresso con \textbf{32 bit}, è \textbf{diviso in 2 parti}
            (rete ed host) ed è utilizzato nei campi Source address e Destination address dei pacchetti IP. È
            importante notare che un indirizzo IP non si riferisce veramente a un host, ma a una \textbf{scheda di
            rete}, perciò quando un host ha due schede di rete, deve avere due indirizzi IP. Comunque, la
            maggior parte degli host ha un’unica scheda di rete e perciò ha un solo indirizzo IP.\\
            
            Un indirizzo IP si esprime in \textbf{notazione decimale a punti}, dove ognuno dei 4 byte è
            rappresentato con un numero che varia tra 0 e 255, del tipo:
            
            \begin{center}
                \textbf{\emph{x.y.z.w}}
            \end{center}
            Un indirizzo IP può essere:
            \begin{itemize}
                \item \textbf{Pubblico}: rilasciato dall’ICANN, indica un indirizzamento di rete destinato a reti \textbf{globali};
                \item \textbf{Privato}: indica un indirizzamento di rete destinato a reti \textbf{locali};
                \item \textbf{Statico}: l’IP dell’host \textbf{non varia} nel tempo;
                \item \textbf{Dinamico}: l’IP dell’host \textbf{varia} nel tempo.
            \end{itemize}
            \subsubsection*{Come si ottiene un indirizzo IP?}
            Un indirizzo IP può essere assegnato automaticamente utilizzando il servizio DHCP (Dynamic
            Host Configuration Protocol) oppure in modo manuale.

            \subsubsection*{Classificazione di Indirizzi IP}
            Per molti decenni, gli indirizzi IP sono stati divisi in cinque categorie: A, B, C, D, E.
            È bene notare che oggi il sistema non è più utilizzato poiché permetteva parecchi sprechi di
            indirizzi IP, sempre più preziosi con l’aumentare della richiesta. Come approfondimento, il
            nuovo sistema attualmente in uso è il \textbf{CIDR (Classless Inter-Domain Routing)}.\\
            
            %da inserire tabella classificazione indirizzi (pagina 16)%
            \begin{tikzpicture}
                \coordinate (A) at (-7,0);
                \coordinate (B) at ( 5,0);
                \draw[<->] (A) -- (B) node[midway,fill=white] {32 bit};
            \end{tikzpicture}
            
            \begin{tabular}{|l|l|l|l|l|l|l|l|l|l|l|l|l|l|l|l|l|l|l|l|l|l|l|l|l|l|l|l|l|l|l|l|}
                \hline
                & & & & & & & & & & & & & & & & & & & & & & & & & & & & & & & \\
                \hline
                \multicolumn{32}{l}{} \\
                \hline
                0 & \multicolumn{7}{|c|}{Rete} & \multicolumn{24}{|c|}{Host} \\
                \hline
                \multicolumn{32}{l}{} \\
                \hline
                \multicolumn{2}{|c|}{10} & \multicolumn{14}{|c|}{Rete} & \multicolumn{16}{|c|}{Host} \\
                \hline
                \multicolumn{32}{l}{} \\
                \hline
                \multicolumn{3}{|c|}{110} & \multicolumn{21}{|c|}{Rete} & \multicolumn{8}{|c|}{Host} \\
                \hline
                \multicolumn{32}{l}{} \\
                \hline
                \multicolumn{4}{|c|}{1110} & \multicolumn{28}{|c|}{Indirizzi multicast} \\
                \hline
                \multicolumn{32}{l}{} \\
                \hline
                \multicolumn{4}{|c|}{1111} & \multicolumn{28}{|c|}{Riservati per usi futuri} \\
                \hline
            \end{tabular}\\
            
            A) Classe utilizzata per le \textbf{WAN} e le \textbf{MAN};\\
            B) Classe utilizzata per le \textbf{MAN} e le \textbf{grosse LAN};\\
            C) Classe utilizzata per le \textbf{LAN};\\
            D) Classe utilizzata per gli \textbf{indirizzi multicast};\\
            E) Classe non commerciale (riservata).

            \subsubsection*{Indirizzi IP Speciali}
            Esistono alcuni indirizzi IP che hanno un significato speciale. Uno schema di essi, è visibile nella
            tabella sottostante.\\

            %Inserire tabella degli indirizzi IP speciali (pagina 17)%
            \begin{tabular}{ll}
                \begin{tabular}{|l|l|l|l|l|l|l|l|l|l|l|l|l|l|l|l|l|l|l|l|l|l|l|l|l|l|l|l|l|l|l|l|}
                    \hline
                    \multicolumn{32}{|l|}{0 0 0 0 0 0 0 0 0 0 0 0 0 0 0 0 0 0 0 0 0 0 0 0 0 0 0 0 0 0 0 0} \\
                    \hline
                    \multicolumn{8}{|c|}{00 ... 00} & \multicolumn{24}{|c|}{Host} \\
                    \hline
                    \multicolumn{32}{|l|}{1 1 1 1 1 1 1 1 1 1 1 1 1 1 1 1 1 1 1 1 1 1 1 1 1 1 1 1 1 1 1 1} \\
                    \hline
                    \multicolumn{8}{|c|}{Rete} & \multicolumn{24}{|c|}{1111 ... 1111} \\
                    \hline
                    \multicolumn{3}{|c|}{127} & \multicolumn{29}{|c|}{(Qualsiasi cosa)} \\
                    \hline
                \end{tabular}
                &
                \begin{tabular}{l}
                    Questo host\\
                    Un host su questa rete\\
                    Trasmissione broadcast sulla rete locale\\
                    Trasmissione broadcast su rete distante\\
                    Loopback\\
                \end{tabular}
            \end{tabular}\newline\newline

            L’indirizzo IP \textbf{0.0.0.0} è utilizzato dagli host al momento del boot. Gli indirizzi IP di tipo C che
            hanno lo \textbf{0 come numero di rete} si riferiscono alla rete corrente. Grazie a questi indirizzi, i
            computer possono utilizzare la rete locale senza conoscerne il numero. L’indirizzo composto
            da \textbf{tutti 1} permette la trasmissione broadcast sulla rete locale, in genere una LAN. Gli indirizzi
            con \textbf{un numero di rete opportuno} e tutti 1 nel campo host, permettono ai computer l’invio di
            pacchetti broadcast a reti LAN distanti collegate a Internet (però molti amministratori di rete
            disattivano questa funzionalità). Infine, tutti gli indirizzi espressi nella forma \textbf{127.xx.yy.zz} sono
            riservati per le prove di \underline{loopback}. I pacchetti diretti all’indirizzo di \underline{loopback} non sono immessi
            nel cavo, ma elaborati localmente e trattati come pacchetti in arrivo.
            
            \subsubsection{Subnetting}
            Può rivelarsi estremamente utile (come in casi in cui una singola rete acquisisce dimensioni
            parecchio elevate) dividere internamente una rete in più parti, facendo in modo che il mondo
            esterno veda ancora una singola rete. Questa pratica è chiamata \textbf{subnetting} e le parti della rete
            sono chiamate \textbf{subnet (sottoreti)}.\\

            Affinché un router implementi le sottoreti e sia in grado di riconoscere il numero di host
            raggiungibili all’interno di ognuna, si definisce la \textbf{maschera di sottorete (subnet mask)}, un
            numero che, preso un indirizzo IP, indica il punto di demarcazione tra i bit di rete e i bit di host.\\
            
            Come l'indirizzo IP, la subnet mask è un \textbf{numero a 32 bit} che si può indicare in notazione
            decimale a punti oppure come un unico numero decimale, indicante il numero di bit della parte
            di indirizzo della rete.\\
            
            Normalmente indirizzo IP e subnet mask sono accoppiati nella forma \textbf{a.b.c.d/m}, dove m è la
            subnet mask in forma decimale, che è anche possibile ottenere in notazione decimale a punti:
            basta infatti porre i primi m bit significativi di un numero a 32 bit pari a 1 ed i rimanenti pari a
            0, per poi raggruppare in ottetti e convertire in decimale.\\

            Effettuando l'AND logico bit a bit tra la subnet mask e l'indirizzo IP, si ottiene l'indirizzo IP della
            sottorete.\\
            
            Si noti che effettuare subnetting, vuol dire conseguentemente \textbf{aumentare} i bit della parte \textbf{rete}
            dell’indirizzo e \textbf{diminuire} i bit della parte \textbf{host}.

            %Inserire tabella sbnetting (pagina 18)%

            \subsubsection{Supernetting}
            Analogamente, esiste il concetto di \textbf{supernetting}, che consiste nell’accorpare più reti fisiche
            primarie diverse in un’unica rete. Fare supernetting può risultare utile, ad esempio, per
            semplificare le informazioni di routing da trasmettere ai router.\\

            Al contrario del subnetting, effettuare supernetting vuol dire conseguentemente \textbf{aumentare} i
            bit della parte \textbf{host} dell’indirizzo e \textbf{diminuire} i bit della parte \textbf{rete}.
            
            \subsubsection{Piano di indirizzamento IP}
            Il piano di indirizzamento IP è un documento che l’amministratore di rete deve scrivere e tenere
            aggiornato per descrivere l’utilizzo del proprio spazio di indirizzamento IP.
            
            \subsubsection{IPv6}
            L’IPv6 è la nuova versione di IP che non dovrebbe mai esaurire i suoi indirizzi IP, risolvere una
            varietà di problemi in IPv4 ed essere più flessibile ed efficiente, permettendo quindi una reale
            connettività globale, eliminando reti o host nascosti.\\
            
            Nello specifico, sono 4 i cambiamenti portati da questo nuovo standard:
            \begin{itemize}
                \item \textbf{Indirizzi da 16 byte} (128 bit), il che si traduce in una scorta praticamente illimitata di
                indirizzi IP;
                \item \textbf{Header semplificato} (7 campi contro 13 di IPv4), che permette di elaborare i pacchetti
                più velocemente;
                \item \textbf{Migliore supporto per le opzioni}, necessario poiché campi prima necessari ora sono
                opzionali;
                \item \textbf{Incremento della sicurezza}, migliorando autenticazione e riservatezza.                
            \end{itemize}
            Di seguito è descritta l’intestazione dell’IPv6.

            %Inserire Tabella ipv6 (pag 19)%
            
            \begin{itemize}
                \item \textbf{Version}: campo di 4 bit che indica la versione IP del datagram (che sarà sempre 6);
                \item \textbf{Traffic class}: utilizzato per distinguere i pacchetti con diversi requisiti di distribuzione
                in tempo reale (simile a Type of Service di IPv4);
                \item \textbf{Flow label}: sebbene il suo utilizzo non sia ancora chiaro, consentirebbe a una sorgente e
                a una destinazione di impostare una pseudo-connessione con particolari proprietà e
                requisiti;
                \item \textbf{Payload length (lunghezza del carico utile)}: numero di byte del campo data;
                \item \textbf{Next header}: è il segreto con cui è stato possibile semplificare l’header dell’IPv4.
                Permette di indicare header estesi (opzionali) che seguono l’intestazione corrente. Se
                però l’intestazione corrente è l’ultima intestazione IP, il campo next header indica il
                gestore del protocollo di trasporto TCP o UDP al quale va passato il pacchetto (come il
                campo Protocol dell’IPv4);
                \item \textbf{Source address e Destination address}: come in IPv4, indicano l’indirizzo IP sorgente e
                l’indirizzo IP di destinazione.
                
            \end{itemize}
            Si noti che i campi riguardanti la frammentazione in IPv6 non sono più presenti, poiché IPv6
            la gestisce in maniera diversa, così come il campo checksum è stato eliminato, perché
            l’elaborazione di questa informazione riduceva enormemente le prestazioni, che con le reti
            sempre più affidabili odierne, risulta quasi totalmente superfluo.

            \subsubsection*{Header estesi IPv6}
            Come accennato prima, IPv6 implementa il concetto di header. Alcuni dei campi mancanti di
            IPv4 sono ancora saltuariamente necessari, perciò IPv6 ha introdotto il concetto di intestazione
            estesa (opzionale), chiamata anche extension header. Queste intestazioni possono essere
            fornite per offrire informazioni aggiuntive, codificate in modo efficiente. Al momento sono
            definiti sei tipi di intestazioni estese:

            \begin{itemize}
                \item \textbf{Hop-by-hop}: utilizzata per le informazioni che tutti i router lungo il percorso devono
                esaminare.
                \item \textbf{Destination options}: utilizzata dai campi che devono essere interpretati dall’host di
                destinazione;
                \item \textbf{Routing}: elenca uno o più router che devono essere visitati lungo la strada;
                \item \textbf{Fragment}: si occupa delle informazioni di frammentazione proprio come fa IPv4;
                \item \textbf{Authentication}: fornisce un meccanismo che garantisce al ricevente di un pacchetto
                l’autenticità del mittente;
                \item \textbf{Encapsulating Security Payload}: permette di proteggere con la crittografia il
                contenuto del pacchetto in modo da renderlo leggibile solo ai destinatari desiderati.
            \end{itemize}
            
            \subsubsection*{Indirizzi IPv6}
            Gli indirizzi IPv6 ovviamente svolgono ancora la stessa funzione degli indirizzi IPv4. Ciò
            nonostante, avendo ora a disposizione 128 bit per essere rappresentati, la loro definizione va
            rivista. Per scrivere gli indirizzi a 16 byte è stata escogitata una nuova notazione. Gli indirizzi
            sono scritti come otto gruppi di quattro cifre esadecimali separate da due punti:
            \begin{center}
                \textbf{a:b:c:d:e:f:g:h}
            \end{center}
            Sono ammesse alcune scritture facilitate:
            \begin{itemize}
                \item È possibile omettere gli zero iniziali di un gruppo, ad esempio 0123 -> 123;
                \item Uno o più gruppi contenenti 16 bit a zero possono essere sostituiti da due punti, ma ciò
                è concesso solo una volta ad indirizzo. Ad esempio:
                \underline{8000:0000:0000:0000:0123:4567:89AB:CDEF} $\rightarrow$ \underline{8000::123:4567:89AB:CDEF};
            \end{itemize}

            Inoltre, gli indirizzi IPv4 possono essere scritti in notazione decimale a punti dopo una coppia
            di due punti, ad esempio \underline{::192.31.20.46}.\\

            Per quanto riguarda la scrittura di un indirizzo IPv6 in un URL, esso deve essere scritto fra
            parentesi quadre. Ad esempio: \url{http://[2001:1:4F3A::206:AE14]:8888/index.html}
            
            \subsubsection*{Tipi di indirizzi IPv6}
            Gli indirizzi IPv6 possono essere di vario tipo: \textbf{Unicast, Multicast, Anycast}.
            
            \subsubsection*{Unicast}
            Uno a uno: è un indirizzo IP che identifica una sola interfaccia di rete e quindi un unico
            destinatario. L’indirizzo IPv6 unicast è diviso in 2 parti: i primi 64 bit identificano il prefisso di
            rete e vengono detti \textbf{Subnet Prefix}, i restanti 64 identificano l’host e vengono detti quindi \textbf{Host
            Identifier}. Gli indirizzi di tipo unicast possono essere:
            \begin{itemize}
                \item \textbf{Unspecified}: assenza di indirizzo, può essere usato nella richiesta iniziale DHCP per
                ottenere un indirizzo (formato: 0:0:0:0:0:0:0:0 oppure ::);
                \item \textbf{Loopback}: identifica l’host stesso, equivalente del 127.0.0.0 in IPv4
                (formato: 0:0:0:0:0:0:0:1 oppure ::1); Per controllare se lo stack IPv6 funziona, è
                possibile eseguire infatti il comando `ping6 ::1`;
                \item \textbf{Link Local}: è uno scoped address, una novità di IPv6, ovvero un indirizzo di rete che è
                valido solo per la comunicazione all’interno di un segmento di rete (link).
                Fornisce ad ogni nodo un indirizzo IPv6 per iniziare le comunicazioni
                (formato: FE80:0:0:0:<interface identifier>);
                \item \textbf{Site Local}: è anch’esso uno scoped address, in questo caso l’ambito dell’indirizzo è un
                \textbf{site} (una rete di link). Non è configurato di default e può essere utilizzato per
                l’indirizzamento privato (format: FEC0:0:0:<subnet id>:<interface id>).                
            \end{itemize}
            
            \subsubsection*{Multicast}
            Uno a molti: è un indirizzo IP che identifica un insieme di interfacce di rete e quindi un insieme
            di destinatari. Si noti che \underline{non esiste il broadcast in IPv6}, ed al suo posto si usa il multicast. Il suo
            formato è FF<flags><scope>::<multicast group>, dove flag può essere 0 (permanente) o 1
            (temporaneo).

            \subsubsection*{Anycast}
            Uno al più vicino: un indirizzo IPv6 anycast è un indirizzo che può corrispondere a un insieme
            di interfacce di rete. Serve per le funzioni di discovery e non sono distinguibili dagli indirizzi
            unicast.
            
        \subsection{Configurazione IP}
        Un host IP può essere configurato automaticamente o staticamente. Della configurazione
        automatica se ne occupa il protocollo \textbf{DHCP (Dynamic Host Configuration protocol)} che,
        oltre a semplificare la fase di configurazione, permette una gestione facilitata della rete: di fatti
        così facendo è possibile cambiare gli indirizzi IP e/o i parametri di configurazione senza
        intervenire sui client, il che torna molto utile nel caso di grandi organizzazioni. Il suo
        funzionamento viene analizzato nei capitoli successivi.
        Per la configurazione statica, occorre specificare le seguenti entità:
        \begin{itemize}
            \item \textbf{Indirizzo IP};
            \item \textbf{Subnet Mask};
            \item \textbf{Default gateway};
            \item Eventualmente \textbf{Indirizzo IP del nameserver} (vedi DNS).
        \end{itemize}       

        \subsection{Internet Multicasting}
        La normale comunicazione IP avviene tra un trasmittente e un ricevente, ma per alcune
        applicazioni è utile che un processo possa comunicare contemporaneamente con un gran
        numero di ricevitori. Si pensi banalmente alla gestione di video-conferenze digitali.\\
        
        IP supporta la trasmissione \textbf{multicast}, in particolare IPv4 lo implementa con gli indirizzi di
        classe D. Ogni indirizzo di classe D identifica un gruppo di host e 28 bit sono utilizzati per
        identificare i gruppi. Il tentativo che viene fatto per trasmettere i dati ai membri del gruppo è
        di tipo \textbf{best-effort}.\\
        
        In un gruppo, gli host possono cambiare dinamicamente, ovvero possono aggiungersi o uscire
        dal gruppo senza limitazioni, ed è possibile definire una chiave di accesso.\\

        Sono supportati due tipi di indirizzi di gruppo: \textbf{permanenti} e \textbf{temporanei}.\\
        
        Un gruppo \textbf{permanente} è sempre presente, non deve essere impostato e possiede un indirizzo
        di gruppo permanente, mentre un gruppo \textbf{temporaneo} deve essere creato prima di poter
        essere utilizzato, mediante entità specializzate chiamate \textbf{multicast agents}. Oltre a provvedere
        alla creazione di nuovi gruppi, i multicast agents sono responsabili dell’invio in Internet dei
        datagram multicast, nel caso un host sia in una rete diversa rispetto a quella degli altri host.

            \subsubsection*{IGMP}
            Circa una volta al minuto, ogni router multicast genera una trasmissione hardware multicast
            (sul livello data link) diretta agli host sulla sua LAN (indirizzo 224.0.0.1) che chiede alle
            macchine di indicare i gruppi di appartenenza dei loro processi. Ogni host risponde
            comunicando tutti gli indirizzi di classe D che gli interessano.\\
            
            Questi pacchetti di interrogazione e risposta utilizzano un protocollo chiamato \textbf{IGMP (Internet
            Group Management Protocol)}, descritto nel documento RFC 1112. Come intuibile, ha solo due
            tipi di pacchetti: interrogazione e risposta, ognuno con un formato semplice prefissato, che
            contiene alcune informazioni di controllo nella prima word del campo carico utile e un indirizzo
            di classe D nella seconda word.\\
            
            Come piccola curiosità, il routing multicast è realizzato mediante le strutture spanning tree.
            