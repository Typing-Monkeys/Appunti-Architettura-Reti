\section{Le Architetture di Rete}
    Esistono due importanti architetture di rete: il modello di riferimento \textbf{OSI} e il modello \textbf{TCP/IP}.
    Anche se i protocolli associati al modello OSI ormai sono in disuso, il modello in sé ha valore
    generale ed è ancora valido, e le caratteristiche discusse per ogni livello sono ancora molto
    importanti. Il modello TCP/IP ha caratteristiche opposte: il modello in sé è poco utilizzabile, ma
    i protocolli sono largamente impiegati; per questo motivo vanno esaminati entrambi in
    dettaglio. A volte s’impara più dai fallimenti che dai successi.
    
    \begin{tabular}{ll}
        \textit{OSI} & \textit{TCP/IP} \\
        \begin{tabular}{|c|c|}
            \hline
            \textbf{Livello 7} & Applicazione \\
            \hline
            \textbf{Livello 6} & Presentazione \\
            \hline
            \textbf{Livello 5} & Sessione \\
            \hline
            \textbf{Livello 4} & Trasporto \\
            \hline
            \textbf{Livello 3} & Rete \\
            \hline
            \textbf{Livello 2} & DataLink \\
            \hline
            \textbf{Livello 1} & Fisico \\
            \hline
        \end{tabular} 
        &
        \begin{tabular}{|c|}
            \hline
            Applicazione \\
            \hline
            Presentazione \\
            \hline
            (non presente) \\
            \hline
            (non presente) \\
            \hline
            Trasporto \\
            \hline
            Internet \\
            \hline
            Host-to-network \\
            \hline
        \end{tabular}
    \end{tabular}

    \subsection{Il modello di riferimento ISO/OSI}
        L’ISO (International Standards Organization) è un organo consulente dell’ONU, che promuove
        lo sviluppo di standardizzazioni nel mondo. È il creatore dell’\textbf{OSI (Open Systems
        Interconnections)}, così chiamato perché riguarda la connessione di sistemi aperti, cioè sistemi
        che sono “aperti” verso la comunicazione con altri. È uno standard \textbf{\textit{de jure}} ("per legge", ovvero
        formale, adottato da qualche organismo di standardizzazione autorizzato, in questo caso ISO).
        È caratterizzato da \textbf{7 livelli}: ogni livello sfrutta i servizi dei livelli inferiori. La comunicazione
        tra livelli adiacenti avviene tramite i \textbf{NAP (Neutral Access Points)}, mentre la comunicazione
        tra entità di livelli diversi avviene tramite il \textbf{SAP (Service Access Point)}. Le operazioni
        specifiche di un livello sono realizzate tramite un insieme di protocolli. Di seguito una breve
        descrizione di ogni livello:

        \begin{itemize}
            \item \textbf{Physical Layer}: si occupa della trasmissione di bit grezzi sul canale di comunicazione.
                È l’insieme di regole che specificano le connessioni elettriche e fisiche tra i dispositivi. Definisce
                la specifica dei cavi e del tipo di segnale elettrico associato ai vari pin.

            \item \textbf{Data Link Layer}: definisce l’accesso al mezzo specificato nel Physical Layer, il formato dei dati
                ed è responsabile dell’invio affidabile delle informazioni, ovvero gestisce tra le altre cose la
                frammentazione dei dati e le procedure di controllo di possibili errori del livello fisico.
                Appartengono a questo livello i protocolli Data Link (DLCP, BSC, HDLC...), sono inoltre presenti
                anche i sottolivelli LLC e MAC e hardware come Network Interface Card (NIC), Hub, Switch (di
                livello 2) e Bridge.
                
            \item \textbf{Network Layer}: si occupa della connessione tra due nodi della rete (detti nodo sorgente e nodo
                destinatario). Gestisce il routing e lo scambio di informazione tra i nodi. I servizi associati a
                questo livello sono legati al movimento dei dati nella rete, inclusi l’indirizzamento, il routing e
                le procedure di controllo dei flussi (vengono definite le raccomandazioni X.25 e X.75).
                A questo livello appartiene il protocollo IP.
                
            \item \textbf{Transport Layer}: è il garante del trasferimento delle informazioni. Analizza il traffico fra i nodi
                controllando gli errori, la sequenza e i fattori di affidabilità dello scambio.
                A questo livello appartengono i protocolli TCP e UDP. È il primo livello \textbf{end-to-end}.
                
            \item \textbf{Session Layer}: regolarizza l’inizio e la fine dei flussi dei dati fra i nodi. Si occupa
                dell’organizzazione del dialogo tra i programmi applicativi.
                
            \item \textbf{Presentation Layer}: si occupa di effettuare trasformazione sui dati compatibilmente con il
                dispositivo di ricezione. Esempi di trasformazione riguardano crittografia e compressione.

            \item \textbf{Application Layer}: è l’ultimo livello, comprende tutti i programmi applicativi che consentono
                l’uso della rete: fa da interfaccia tra rete e utente. Esempi di funzioni svolte da questo livello
                possono essere: terminale virtuale, trasferimento di file, posta elettronica, condivisione di
                risorse e accesso a database.
        \end{itemize}

    \subsection{Il modello di riferimento TCP/IP}
        Il \textbf{TCP/IP} è così chiamato per ricordare i suoi protocolli più importanti: il Transmission Control
        Protocol (TCP) e l'Internet Protocol (IP). È uno standard \textbf{\textit{de facto}} ("dalla realtà", ovvero
        stabilito senza piani formali). Di seguito una breve descrizione di ogni livello:
        
        \begin{itemize}
            \item \textbf{Host-to-network Layer}: “il grande vuoto”, nel modello di riferimento TCP/IP non viene
                specificato cosa accade in questo territorio, si limita a segnalare che l’host deve collegarsi alla
                rete usando qualche protocollo che gli permetta di spedire pacchetti IP. Questo protocollo però
                non è definito e varia da host a host.
                
            \item \textbf{Internet Layer}: il suo scopo è quello di consentire agli host di mandare pacchetti in qualsiasi
                rete, e farli viaggiare in modo indipendente l’uno dall’altro fino alla destinazione (che magari è
                su una rete diversa). È importante notare che viene definito il protocollo \textbf{IP}.
            
            \item \textbf{Transport Layer}: è progettato per consentire la comunicazione tra entità pari degli host
                sorgente e destinazione, come nel livello trasporto OSI. Sono definiti due protocolli di trasporto
                end-to-end: \textbf{TCP} e \textbf{UDP}.

            \item \textbf{Application Layer}: nel modello TCP/IP non ci sono sessione e presentazione, poiché si è notato
                che nella maggior parte dei casi sono inutili. Come nel modello OSI, il livello Applicazione
                contiene un gran numero di programmi che si interfacciano con l’utente.
        \end{itemize}