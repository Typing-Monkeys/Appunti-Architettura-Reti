\section{Servizi di Rete}

    I livelli sotto lo strato applicazione forniscono il trasporto affidabile, ma non svolgono alcun
    lavoro per gli utenti. Di seguito vengono presentate alcune applicazioni concrete delle reti.

    \subsection{Telnet}

        \textbf{Telnet} è un protocollo di rete che ha come obiettivo il fornire un supporto per le comunicazioni
        sufficientemente \textbf{generalizzato}, \textbf{bidirezionale} ed \textbf{orientato ai byte}. È un servizio che
        permette di accedere in remoto ad una macchina, tramite emulazione di terminale attraverso
        la rete, basandosi sul protocollo \textbf{TCP} e sul paradigma client/server. Di solito a tale servizio è
        assegnata la \textbf{porta 23}, ed è possibile utilizzare un programma Telnet per stabilire una
        connessione interattiva ad un qualche altro servizio del server. Si prenda come esempio il
        collegamento alla porta 25, sulla quale si trova un server SMTP (utilizzato per la posta ed
        analizzato in seguito).\\

        Telnet si basa su \textbf{3 aspetti}:

        \begin{itemize}
            \item \textbf{NVT (Network Virtual Terminal)}: è un terminale virtuale con caratteristiche generali. Ogni server o client traduce i propri controlli nativi in quelli del NVT;
            \item \textbf{Opzioni negoziate}: tale negoziazione avviene tra client e server per aumentare le funzionalità della sessione Telnet da aprire;
            \item \textbf{Viste simmetriche}: degli strumenti per evitare che la negoziazione delle opzioni generi
            cicli di opzioni senza fine (errata interpretazione).
        \end{itemize}

        Il problema principale di Telnet riguarda la sicurezza: è di fatti un protocollo non criptato,
        quindi password, nome utente e tutte le altre informazioni sono inviate in chiaro. A questa
        problematica pone rimedio \textbf{SSH} (analizzato in seguito).

    \subsection{Comandi r}

        I \textbf{comandi r} sono progettati per sistemi BSD Unix e forniscono delle facilitazioni specifiche per Unix.
        Di seguito i comandi più usati:

        \begin{itemize}
            \item \textbf{rlogin (remote login)}: permette di amministrare una serie di macchine autenticandosi
            una sola volta (e quindi non richiedendo altre volte la password). È una soluzione molto
            comoda che facilita le configurazioni di ambienti distribuiti, ma porta ovvie
            problematiche di sicurezza;
            
            \item \textbf{rsh (remote shell)}: consente l’esecuzione remota di un singolo comando.
        \end{itemize}

        Anche in questi comandi non vi è alcuna forma di crittografia e pertanto sono stati anch’essi rimpiazzati da \textbf{SSH}.

    \subsection{FTP}        
        \textbf{FTP (File Transfer Protocol)} è un protocollo basato su TCP ed è impiegato per il trasferimento 
        di file in rete. Di solito a tale servizio è assegnata la \textbf{porta 21}.\\

        A differenza di altri protocolli, FTP utilizza 2 connessioni separate per gestire comandi e dati:

        \begin{itemize}
            \item \textbf{PI (Protocol Interpreter)}: si occupa di trasmettere comandi fra il client e il server.
            Tramite esso inoltre si dà inizio al processo FTP;
            \item \textbf{DTP (Data Transfer Process)}: si occupa del trasferimento vero e proprio dei dati tra un
            client e un server. Durante l’esecuzione, client e server si scambiano i ruoli
            continuamente.
        \end{itemize}

        È bene notare che ad oggi, gran parte delle richieste FTP non avviene da linea di comando,
        ma da browser, ad esempio specificando un URL.\\

        Normalmente, FTP richiede l’autenticazione del client tramite nome utente e password,
        tuttavia sono state pensate delle \textbf{sessioni anonime}, che permettono l’utilizzo del protocollo in
        sola lettura, il che evita l’upload sul server di dati non autorizzati. Per usare una sessione
        anonima, basta usare anonymous come username e come password qualsiasi stringa, come ad
        esempio l’indirizzo e-mail o semplicemente la parola "guest".

        Nelle versioni iniziali, FTP prevedeva il trasferimento in chiaro delle password. In seguito, con 
        RFC 2228, sono stati introdotti nuovi comandi per aumentare la sicurezza:

        \begin{itemize}
            \item \textbf{AUTH}: meccanismo utilizzato per l’autenticazione da utilizzare;
            \item \textbf{PORT}: il livello di sicurezza che verrà usato. Ce ne sono diversi: \begin{itemize}
                \item \textbf{Clear}: trasmissione in chiaro;
                \item \textbf{Safety}: richiesta la verifica sull’integrità dei dati, si usa il comando \textbf{MIC};
                \item \textbf{Confidential}: trasmissione cifrata, si usa il comando \textbf{CONF};
                \item \textbf{Private}: trasmissione cifrata e richiesta la verifica sull’integrità dei dati, si usa il comando \textbf{ENC}.
            \end{itemize}
        \end{itemize}

        Nonostante ciò, anche FTP è stato inglobato in SSH per aumentare ancor di più la sicurezza.\\

        Un’alternativa più leggera all’FTP standard è \textbf{TFTP (Trivial FTP)}, che sfrutta UDP al posto di TCP: 
        pensata per l’ambito LAN, è spesso utilizzata dalle macchine diskless per ottenere l’immagine del sistema operativo.

    \subsection{SSH}

        \textbf{SSH (Secure Shell)} è un protocollo per comunicazioni di rete sicure progettato per essere
        relativamente semplice ed economico da implementare, utilizzando la crittografia asimmetrica.\\

        La versione iniziale (SSH1) era focalizzata sulla fornitura di una struttura di accesso remoto
        sicura per sostituire Telnet e altri schemi di accesso remoto che non fornivano sicurezza. Nella
        nuova versione (SSH2), vengono corretti una serie di difetti di sicurezza dello schema originale
        e migliorate performance e flessibilità, che comprende anche l’implementazione di \textbf{SFTP
        (Secure File Transfer Protocol)}.\\

        Le applicazioni client e server SSH sono ad oggi ampiamente disponibili per la maggior parte
        dei sistemi operativi ed è diventato il metodo di scelta principale per l'accesso remoto ad un
        server, rimpiazzando Telnet, i comandi R ed FTP.\\

        SSH è organizzato in tre protocolli che in genere vengono eseguiti su TCP:
        %inserire tabella SSH pagina 39%

        \begin{itemize}
            \item \textbf{Transport Layer Protocol}: definisce l'autenticazione del server, lo scambio delle chiavi,
            la cifratura, la compressione (opzionale) e il controllo d'integrità dei pacchetti ;
            \item \textbf{User Authentication Protocol}: definisce l'autenticazione dell'utente, che può avvenire
            in 3 metodi: utilizzando la public key, la password o l'hostbased;
            \item \textbf{Connection Protocol}: gestisce funzionalità quali l'instaurazione di terminali interattivi,
            esecuzione di comandi remoti e l’inoltro di connessioni e di applicazioni grafiche X11,
            utilizzando canali multipli di comunicazione passanti per lo stesso tunnel criptato del
            Transport Layer.
        \end{itemize}

        I livelli interni al protocollo SSH coprono gli ultimi 3 livelli della pila OSI.\\

        Sono resi disponibili diversi applicativi che utilizzano SSH. Tra questi, \textbf{OpenSSH} è un
        insieme di programmi open source, che rendono disponibili sessioni crittografate di
        comunicazione in una rete. Tra i programmi più utilizzati, sono presenti:

        \begin{itemize}
            \item \textbf{\emph{ssh}}, che sostituisce rlogin e Telnet;
            \item \textbf{\emph{scp}}, che sostituisce rcp;
            \item \textbf{\emph{sftp}}, che sostituisce ftp;
            \item \textbf{\emph{ssh-add}}, \textbf{\emph{ssh-agent}}, e \textbf{\emph{ssh-keygen}}, una serie di programmi per la generazione e la
            gestione delle chiavi;
            \item i nomi dei processi demoni rispettivamente per server SSH e server SFTP sono \textbf{\emph{sshd}} e
            \textbf{\emph{sftp-server}}.
        \end{itemize}

    \subsection{DNS}

        \emph{Il DNS richiederebbe una spiegazione sul funzionamento dei file di configurazione, ma per
        una migliore comprensione e focalizzazione è stata tralasciata.}\\

        Anche se i programmi possono teoricamente fare riferimento a host tramite i loro indirizzi IP,
        per le persone sono difficili da ricordare. Inoltre, come detto, gli indirizzi IP possono cambiare.
        Per questo motivo è stato inventato il \textbf{DNS (Domain Name System)}, un applicativo che effettua
        la \textbf{risoluzione} di nomi di host in indirizzi IP, ovvero la traduzione di un nome sottoforma di
        stringa in indirizzo IP. È anche possibile effettuare una \textbf{risoluzione inversa}.\\

        Le caratteristiche più importanti del DNS sono 2:
        \begin{itemize}
            \item uno \textbf{schema di denominazione gerarchico basato su dominio};
            \item un \textbf{sistema di database distribuito} per l’implementazione del suddetto schema.
        \end{itemize}

        In breve, DNS viene utilizzato come segue: per associare un nome ad un indirizzo IP, un
        programma applicativo chiama una procedura di libreria chiamata \textbf{resolver (risolutore)},
        passando il nome come parametro. Il risolutore invia quindi un pacchetto UDP a un server DNS
        locale, che quindi cerca il nome e restituisce l’indirizzo IP al risolutore, che a sua volta lo
        restituisce al chiamante.

        \subsubsection{Lo spazio dei nomi DNS}

            %Aggiungere immagine Domini pagina 40%

            Il DNS è organizzato ad albero, la cui radice è indicata come “.” e ogni nodo al di sotto 
            di esso è detto \textbf{dominio}. Ogni dominio può essere partizionato ovviamente in \textbf{sottodomini},
             che possono essere a loro volta divisi e così via. Le foglie rappresentano i domini che non hanno sottodomini, che possono sia
            rappresentare un singolo host che rappresentarne migliaia (come nel caso di una società).\\

            Si noti che i nomi di dominio sono \textbf{case insensitive}.\\

            Il primo livello dell’albero è l’insieme dei \textbf{domini di primo livello (}o \textbf{TLD, Top Level Domain)},
            assegnati da \textbf{ICANN}, e ne esistono di 3 tipi:

            \begin{itemize}
                \item \textbf{gTLD (generic TLD)}: domini di primo livello generici, usati da particolari classi di
                organizzazioni (ad esempio si usa \textbf{.com} per organizzazioni commerciali);
                \item \textbf{ccTLD (country-code TLD)}: domini di primo livello nazionali, usati da uno stato o una
                dipendenza territoriale (ad esempio \textbf{.it} per domini italiani);
                \item \textbf{Infrastrutturali}: l’unico è \textbf{.arpa}, usato nella risoluzione inversa dei nomi.
            \end{itemize}

            I \textbf{domini di secondo livello (o SLD, Second Level Domain)}, in genere, appartengono alle
            organizzazioni che li hanno registrati e comprendono il loro nome e il dominio di primo livello
            separati da un punto. Tutti i \textbf{sottodomini} che seguono da destra verso sinistra il SLD vengono
            creati per rendere la gestione del DNS modulare.\\

            Si prenda come esempio \textbf{dmi.unipg.it}:
            \begin{itemize}
                \item il TLD è \textbf{.it}, che sta ad indicare la gerarchia italiana;
                \item il SLD è \textbf{.unipg.it};
                \item il sottodominio è \textbf{.dmi}.
            \end{itemize}

            Al contrario di quanto avviene con gli indirizzi IP, si noti che la parte più importante di un nome
            è la prima partendo da destra (il TLD).\\

            Ottenere un dominio di secondo livello è semplice: basta contattare un \textbf{registrar} (autorità di
            registrazione dei domini) per il dominio di primo livello corrispondente (ad esempio \textbf{.it}), per
            controllare se il nome desiderato è disponibile. Se non vi sono problemi, il richiedente paga una
            tariffa annuale e ottiene il nome.\\

            Ad ogni dominio, che sia rappresentato da un singolo host o da un dominio di primo livello, può
            essere associato un insieme di \textbf{Resource Record (RR, record delle risorse)}.\\

            Un \textbf{RR} è una tupla del database di un DNS Server che contiene la mappatura tra indirizzo IP
            (supportata anche la v6) e nome. Ve ne sono di diversi tipi, ma per sinteticità sono stati omessi.

        \subsubsection{Risoluzione}
            Come anticipato, la traduzione di un nome in un indirizzo IP è detta \textbf{risoluzione}. Essa può
            essere di due tipi:
            
            \begin{itemize}
                \item \textbf{Statica}: la mappatura tra indirizzo IP e nome viene stabilita permanentemente
                mediante una \textbf{host table}, memorizzata in un semplice file ASCII (\textbf{/etc/hosts});
                \item \textbf{Dinamica}: la mappatura tra indirizzo IP e nome viene stabilita ad ogni avvio dell’host,
                effettuando richieste a server DNS. Permette ad un nome di essere sempre associato
                all’indirizzo IP di uno stesso host, anche se l’indirizzo cambia nel tempo.
            \end{itemize}

        \subsubsection{BIND}
            \textbf{BIND (Berkeley Internet Name Domain)} è l’implementazione più comune del DNS su
            ambiente Unix, ed è composto da \textbf{una parte client (il resolver)} ed \textbf{una parte server (named)}
            che comunicano tramite la porta 53.\\

            \textbf{Resolver} è una libreria di funzioni che permette di generare e inviare al server (tramite UDP)
            le richieste di informazioni sui nomi dei domini; \textbf{Named} è un processo demone in grado di
            servire le richieste del resolver e rispondere (tramite TCP).\\

            \textbf{BIND} può essere configurato come:
            
            \begin{itemize}
                \item \textbf{Caching-only}: il server reindirizza ogni richiesta del resolver ad altri server e
                memorizza le risposte in una cache locale;
                \item \textbf{Primary}: il server è gestore di informazioni riguardanti specifici domini. Legge le
                informazioni da appositi file configurati dall’amministratore, detti \textbf{\emph{zone file}};
                \item \textbf{Secondary}: il server scarica gli zone file dal primary server e li memorizza localmente
                in appositi file detti \textbf{\emph{zone file transfer}}.
            \end{itemize}

            Per configurare il resolver, il file relativo è \textbf{‘/etc/resolv.conf’}.\\
            Per configurare il \textbf{named}, i file relativi sono \textbf{‘/etc/named.conf’}, \textbf{‘/etc/named.ca’},
            \textbf{‘/named.local’}, \textbf{‘/etc/named.hosts’}, \textbf{‘/etc/named.rev’}.
    
    \subsection{NIS}

        \textbf{NIS (Network Information Service)} è un servizio (spesso utilizzato nel contesto di
        applicazioni parallele e distribuite) che permette di definire delle risorse di amministrazione
        comuni ad un insieme di host, permettendo un controllo centralizzato e la condivisione
        automatica di risorse. Così facendo, un utente può utilizzare host differenti mantenendo gli
        stessi username, password, cartella home e permessi.\\

        Il NIS non utilizza i file amministrativi (come ad esempio /etc/passwd e /etc/group) così come
        sono, ma ne crea una copia: queste copie sono denominate \textbf{NIS map}. Le \textbf{NIS map} sono dei
        database in cui si memorizzano solo coppie di dati (chiave-valore) e vengono memorizzate nel
        server principale, che le rende disponibili ai client tramite il processo \textbf{ypserv}. I client possono
        aggiornare le loro informazioni ricevendo i database tramite il processo demone \textbf{ypbind}.
        Generalmente, le informazioni NIS sono memorizzate nella directory \textbf{/var/yp}.\\

        Quando si attiva il NIS, si consiglia di non usare più i vecchi comandi amministrativi per
        cambiare password/permessi o quant’altro, perché il NIS non si accorge (autonomamente) dei
        cambiamenti apportati ai file amministrativi tradizionali. Bisogna invece utilizzare i comandi
        specifici del NIS.

    \subsection{NFS}

        \textbf{NFS (Network File System)} è un protocollo che permette la condivisione di directory e file su
        una rete. L’effetto finale che ottiene un utente è poter accedere ed utilizzare file memorizzati su
        sistemi remoti come se fossero locali.\\

        Lato \textbf{client}, l’inserimento nel proprio filesystem di una directory situata in un host remoto
        viene detto \textbf{mounting}, realizzata col comando \textbf{\emph{mount}}. Il processo demone che gestisce l’I/O è
        \textbf{biod}.\\

        Lato \textbf{server}, la condivisione di una directory locale ad host specifici è detta \textbf{sharing} e viene
        realizzata col comando \textbf{\emph{export}}, configurabile tramite il file \textbf{/etc/exports} dove è possibile
        elencare directory da esportare con relativi host che hanno l’accesso. Il processo demone che
        gestisce le richieste NFS è \textbf{nsfd}.\\

        Gli sviluppatori hanno implementato NFS creando 3 software indipendenti, compreso \textbf{NFS}.
        Gli altri due sono \textbf{RPC} e \textbf{XDR}.\\

        \textbf{RPC (Remote Procedure Call)} consente di codificare e decodificare le richieste tra i client ed
        i server, permettendo di accedere ai file system remoti con le stesse procedure che utilizzano
        per accedere ai file locali.\\

        \textbf{XDR (eXternal Data Representation)} consente di scambiare dati tra macchine con
        architettura eterogenea fornendo una rappresentazione dei dati indipendente dalla macchina.
        Grazie ad esso, non bisogna preoccuparsi della conversione tra le diverse rappresentazioni dei
        dati a livello hardware.
    
    \subsection{SNMP}

        \textbf{SNMP (Simple Network Management Protocol)} è ad oggi il più potente e diffuso protocollo
        di \textbf{gestione di reti}, sistemi e applicazioni, introducendo una semplice \textbf{architettura} chiamata
        \textbf{INMF (Internet Network Management Framework)}.\\

        SNMP definisce la modalità di scambio di informazioni tra apparecchiature di rete, consentendo
        inoltre agli amministratori di tenere sotto controllo le prestazioni della rete e accorgersi in
        tempo reale di malfunzionamenti della stessa.\\

        Prevede 2 \textbf{ruoli}:
        
        \begin{itemize}
            \item \textbf{Agent}: moduli software che risiedono sui dispositivi da gestire (come \emph{host, stampanti,
            router, switch, hub...});
            \item \textbf{Manager}: uno o più host che possono interrogare e inviare comandi agli agent, tramite
            UDP, sulle porte 161 e 162.
        \end{itemize}

        SNMP è il protocollo di gestione e definisce le modalità d’interazione tra il manager e gli agent.\\

        Le informazioni trattate dagli agent sono dette \textbf{managed objects} (in italiano vengono detti
        semplicemente \textbf{oggetti}) e vengono raccolte in un database chiamato \textbf{MIB (Management
        Information Base)}. Un MIB ha una \textbf{struttura ad albero}:

        \begin{itemize}
            \item i \textbf{nodi} sono organizzazioni e sotto-organizzazioni rappresentate da una etichetta ed un
            numero. Il nodo più importante è \textbf{mib-2}, dove vengono raccolte le variabili usate da
            SNMP attualmente raggruppabili in 12 categorie (altre verranno aggiunte);
            \item le \textbf{foglie} rappresentano i managed objects. Ogni oggetto all’interno del MIB viene
            identificato mediante un \textbf{OID (Object ID)}, ovvero la sequenza di numeri dei nodi
            attraversati nel cammino dalla radice alla foglia, intervallati da punti.
        \end{itemize}

        Le strutture dati usate nel MIB sono definite sotto forma di regole nello \textbf{SMI (Structure of
        Management Information)}, specificate con lo standard ISO chiamato ASN.1 (Abstract Syntax
        Notation One), un linguaggio di definizione di oggetti. In particolare vengono definiti nome,
        sintassi e codifica.\\

        SNMP conta finora 3 versioni. La \textbf{terza versione} del protocollo rimedia alle carenze in merito
        a \textbf{sicurezza} e \textbf{privacy} delle versioni precedenti, tramite un sistema di autenticazione e
        strumenti di controllo agli accessi.\\

        Il software open source più popolare per la gestione delle reti utilizzando SNMP è \textbf{nagios}.
        
    \subsection{DHCP}

        \textbf{DHCP (Dynamic Host Configuration Protocol)} permette agli host di una rete locale di
        ricevere, ad ogni loro richiesta di accesso ad una rete, la configurazione IP necessaria per
        stabilire una connessione ed operare. Utilizza \textbf{UDP} sulla \textbf{porta} 67 (lato server) e 68 (lato client).\\

        Come RARP e \textbf{BOOTP (Bootstrap Protocol}, predecessore di DHCP, ormai in disuso perché
        permetteva solo la configurazione manuale), anche DHCP utilizza l’architettura \textbf{client/server}:
        il \textbf{server} (un host, molto spesso un \textbf{router}) provvede all’assegnazione degli indirizzi IP ai \textbf{client}
        che li richiedono.\\

        Poiché il server DHCP potrebbe non essere raggiunto dalle trasmissioni broadcast (i router non
        fanno passare i pacchetti broadcast), è necessario installare in ogni LAN un \textbf{DHCP relay agent
        (agente di inoltro DHCP)}, che si occupa di intercettare e rigirare in modalità \textbf{\emph{unicast}} al server
        DHCP (che può trovarsi anche in una rete distante) le trasmissioni DHCP intercettate. È chiaro
        che il relay agent ha bisogno di un’unica informazione: l’indirizzo IP del server DHCP.\\

        DHCP presenta 3 opzioni di configurazione:

        \begin{itemize}
            \item \textbf{Automatica}: i client ottengono un indirizzo IP permanente;
            \item \textbf{Dinamica}: ad ogni nuova connessione i client ottengono un indirizzo IP, il quale ha un
            tempo di validità (detto \textbf{lease}) al cui termine ritorna tra gli indirizzi disponibili. Ciò
            permette di riutilizzare indirizzi non più in uso dai client;
            \item \textbf{Manuale}: i client ottengono un indirizzo IP assegnato dall’amministratore e il DHCP è
            utilizzato semplicemente per comunicare l’indirizzo scelto.
        \end{itemize}

        Il processo di assegnamento degli indirizzi si divide in 4 fasi:

        \begin{itemize}
            \item \textbf{Discovering}: il client invia in modalità broadcast un pacchetto \textbf{DHCP Discover}, per
            richiedere l’assegnamento di un indirizzo. Si noti che in questa fase il client è ancora
            sprovvisto di indirizzo IP e il pacchetto avrà come mittente 0.0.0.0;
            \item \textbf{Offering}: i server che ricevono la richiesta, rispondono (se hanno indirizzi liberi a
            disposizione) con \textbf{DHCP Offer}, in cui propongono una configurazione IP al client;
            \item \textbf{Requesting}: una volta ricevute le offerte dai server, il client le valuta e risponde in
            modalità broadcast con \textbf{DHCP Request}, in cui comunica quale server ha scelto;
            \item \textbf{Acknowledgment}: se l’assegnamento è avvenuto con successo, il server risponde al
            client con \textbf{DHCP ACK}, altrimenti con \textbf{DHCP NACK (Negative Acknowledgment)}.
        \end{itemize}

        Nei sistemi Unix, una possibile implementazione Open Source del DHCP è rappresentata da
        \textbf{dhcpd}, la cui configurazione viene eseguita mediante un file di testo chiamato \textbf{dhcpd.conf}.

    \subsection{NAT}
        
        Il problema dell’esaurimento degli indirizzi IP non è un problema teorico che potrebbe
        presentarsi in un lontano futuro: sta accadendo proprio qui e ora. La soluzione a lungo termine
        è rappresentata da IPv6, ma la transizione sta procedendo lentamente e si è resa necessaria una
        soluzione attuabile in tempi brevi.\\
        La soluzione adottata si chiama \textbf{NAT (Network Address Translation)}. L’idea di base di NAT è
        assegnare ad ogni azienda un singolo indirizzo IP (o, al massimo, un piccolo numero di indirizzi)
        per il traffico di Internet. Dentro l’azienda, ogni host riceve un indirizzo IP privato e locale.
        Nessuno di questi indirizzi potrà apparire su Internet, infatti quando un pacchetto lascia
        l’azienda e va verso Internet, viene effettuata una \textbf{traduzione di indirizzo} (di solito dal router).