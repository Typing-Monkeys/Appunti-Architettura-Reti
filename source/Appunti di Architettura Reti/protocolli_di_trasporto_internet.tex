\section{Protocolli Di Trasporto Internet}
    Il livello trasporto non è uno strato qualsiasi: è il cuore dell’intera gerarchia di protocolli.\\
    
    Il suo compito è fornire il trasporto dei dati, affidabile ed efficiente in termini di costi, dal
    computer di origine a quello di destinazione, indipendentemente dalla rete o dalle reti fisiche
    effettivamente utilizzate. Senza il livello trasporto, l’intero concetto di protocolli a strati
    avrebbe poco senso.\\
    
    Internet possiede \textbf{due protocolli principali} nel livello trasporto: un protocollo senza
    connessione e uno orientato alla connessione, rispettivamente UDP e TCP.
        \subsection{UDP}
        UDP (User Datagram Protocol, ovvero protocollo per datagrammi utente) offre alle
        applicazioni un modo per inviare datagrammi IP incapsulati senza dover stabilire una
        connessione. Pur essendo di conseguenza poco affidabile, esso è impiegato in applicazioni
        time-sensitive, dove è richiesta velocità elevata: un esempio è lo streaming di video e audio.\\
        
        UDP trasmette segmenti costituiti da un’intestazione di 8 byte seguita dal carico utile.
        % Pacchetto UDP (pagina 33)%
        \begin{itemize}
            \item Le \textbf{due porte} servono per identificare i punti finali all’interno dei computer di origine e
            destinazione. Quando arriva un pacchetto UDP, il suo carico utile è consegnato al
            processo associato alla porta di destinazione. Senza le porte, il livello trasporto non
            saprebbe cosa farsene del pacchetto. È utile notare che l’inclusione del campo source
            port è utile nel caso si debba inviare una risposta all’origine;
            \item Il campo \textbf{UDP length} include l’intestazione di 8 byte e i dati;
            \item \textbf{UDP checksum} è utilizzato per verificare l’integrità dei dati. È facoltativo, anche se
            disattivarlo è insensato a meno che la qualità dei dati non sia importante (per esempio
            nel caso di voce digitalizzata).
            
        \end{itemize}
        Il protocollo non si occupa del controllo di flusso, del controllo degli errori o della
        ritrasmissione dopo la ricezione di un segmento errato. Questi compiti sono invece lasciati ai
        processi utente.\\

        Un’altra funzionalità importante di UDP è il \textbf{multiplexing} di più processi utilizzando le porte,
        ovvero la trasmissione (praticamente simultanea) di due o più messaggi da parte di due o più
        processi attraverso la stessa linea.
        
        \subsection{TCP}
        \textit{Il TCP è un argomento abbastanza profondo e complicato e quindi, come anticipato, questo
        argomento è uno di quelli trattati in maniera semplificata.}\\

        \textbf{TCP (Transmission Control Protocol}, ovvero protocollo di controllo della trasmissione) è un
        protocollo full-duplex con connessione. Affidabile e di applicabilità generale, è progettato per
        adattarsi dinamicamente alle peculiarità delle diverse reti che compongono Internet,
        contribuendo ad isolare le applicazioni dai dettagli di networking.\\
        
        In una connessione TCP, ogni byte ha un proprio numero di sequenza a 32 bit. Questa è una
        funzionalità vitale di TCP.\\
        
        Ogni computer che supporta TCP dispone di \textbf{un’entità di trasporto TCP}, che può essere una
        procedura di libreria, un processo utente o una parte del kernel, ma in tutti i casi gestisce i flussi
        TCP e le interfacce verso lo strato IP. Le entità TCP di invio e ricezione frazionano i dati in
        blocchi, detti segmenti. Un \textbf{segmento TCP} consiste di un’intestazione fissa di 20 byte (più una
        parte facoltativa) seguita da 0 o più byte di dati, ed è il software TCP a decidere la dimensione
        dei segmenti. Tuttavia, ogni rete ha una \textbf{MTU (Maximum Transfer Unit}, ovvero unità di
        trasferimento massima) e ogni segmento deve essere contenuto nella MTU, generalmente 1500
        byte.
        %Inserire tabella TCP (pagina 34)%
        \begin{itemize}
            \item \textbf{Source port e Destination Port}: estremi locali della connessione. Una porta e
            l’indirizzo IP del suo host formano un end point univoco e identificano quindi la
            connessione. Questa coppia viene chiamata socket e consente il multiplexing, facendo
            sì che una certa porta possa essere condivisa da più host;
            \item \textbf{Sequence number e acknowledgement number}: il primo specifica la posizione del
            primo byte contenuto nel campo dati all’interno dello stream di byte che il trasmettitore
            del segmento invia; il secondo specifica la posizione del prossimo byte aspettato
            all’interno del segmento inviato dal ricevitore del presente segmento;
            \item \textbf{TCP header length}: numero di parole di 32 bit contenute nell’header (necessario
            perché options ha lunghezza variabile);
            \item 6 flags: 
                \begin{enumerate}
                    \item \textbf{URG}: 1 se si usa il campo urgent pointer;
                    \item \textbf{ACK}: 1 se l’ack number è valido (cioè se si trasporta un ack);
                    \item \textbf{PSH}: 1 se si vuole che i dati vengano consegnati all’applicazione all’arrivo e non
                    archiviarli in un buffer per poi trasmettere un buffer completo (come potrebbe
                    fare per migliorare l’efficienza);
                    \item \textbf{RST}: 1 se si richiede un reset della connessione (per problemi);
                    \item \textbf{SYN}: usato per stabilire le connessioni:
                        \begin{itemize}
                            \item \textbf{SYN=1, ACK=0} $\rightarrow$ richiesta connessione;
                            \item \textbf{SYN=1, ACK=1} $\rightarrow$ accettata connessione;
                            \item \textbf{SYN=0, ACK=1} $\rightarrow$ ricevuta la conferma di accettata connessione.
                        \end{itemize}
                    \item \textbf{FIN}: usato per rilasciare una connessione: non ci sono altri dati da trasmettere.
                \end{enumerate}
            \item \textbf{Window size}: indica quanti byte possono essere inviati a partire da quello che ha
            ricevuto acknowledgement. Utilizzato per la sliding window, spiegata in seguito;
            \item \textbf{Checksum}: simile a quello di UDP, verifica che sia l’header che i dati del segmento siano
            arrivati a destinazione senza errori;
            \item \textbf{Urgent pointer}: puntatore ai dati urgenti;
            \item \textbf{Options}: contiene caratteristiche extra non disponibili nella normale intestazione.
                
        \end{itemize}
        Le connessioni in TCP vengono stabilite mediante l’\textbf{handshake a tre vie}. In TCP, client e server
        tengono traccia di ciò che hanno inviato utilizzando un numero di sequenza, detto \textbf{ISN (Initial
        Sequence Number)}, un campo di 32 bit inizializzato casualmente che viene inviato nei campi
        acknowledgement number e sequence number.
        Per stabilire una connessione, sono necessari 3 passaggi:
        \begin{enumerate}
            \item L’host client invia al server un pacchetto preliminare \textbf{(SYN), chiedendo di
            sincronizzarsi} con il proprio ISN;
            \item  L’host server riceve il pacchetto SYN, \textbf{riconoscendo} l’ISN del client e predisponendo il
            buffer che accoglierà i segmenti che gli verranno inviati. Il server si conserva quindi l’ISN
            del client aumentato di 1, poiché il prossimo pacchetto che gli manderà il client avrà
            questo ISN. Quindi il server invia al client un altro pacchetto \textbf{(SYNACK), chiedendo di
            sincronizzarsi} con il proprio ISN;
            \item Ricevuto il SYNACK, anche il client \textbf{riconosce} l’ISN del server e alloca un buffer per la
            ricezione dei segmenti, conservando l’ISN del server aumentato di 1 (come ha fatto il
            server). Invia quindi un ultimo pacchetto (\textbf{ACK}), che sta ad indicare che la connessione
            è ormai instaurata con successo. 
                        
        \end{enumerate}

        I nomi dei pacchetti spediti nei passaggi dell’handshake sono dati per ricordarsi delle flag poste
        a 1. Si noti che l’inizializzazione casuale dell’ISN è necessaria per diversi motivi, tra cui la
        sicurezza: un utente terzo potrebbe infatti effettuare spoofing di una delle due parti (ovvero
        spacciarsi per una delle due parti utilizzando il suo stesso indirizzo IP) e stabilire una
        connessione con l’altra semplicemente predicendo l’ISN.\\
        
        Come anticipato durante la spiegazione dell’intestazione, il protocollo di base utilizzato dalle
        entità TCP è il protocollo \textbf{sliding window}, utilizzato per il controllo di flusso. Grazie al
        protocollo sliding window, \underline{il mittente può inviare più di un segmento anche se non ha ancora
        ricevuto il riscontro per il primo frame inviato}. Comunque, è garantito che nessun segmento
        andrà perso e che i frame arriveranno nell’ordine corretto. Il protocollo ha questo nome
        perché utilizza 2 finestre (ovvero 2 intervalli) che vengono modificati durante l’algoritmo,
        avanzando sui numeri di segmenti spediti e ricevuti:
        \begin{itemize}
            \item Una finestra è dedicata al \textbf{mittente}: contiene i numeri dei \textbf{segmenti già spediti} ma per
            cui non si è ancora ricevuta una conferma (\textbf{no ACK});
            \item l’altra finestra è per il \textbf{ricevente}: contiene i numeri dei \textbf{segmenti che può ricevere
        senza doverli subito confermare}.
        \end{itemize}
        Quando un mittente invia un segmento, viene fatto partire un timer. Una volta arrivato a
        destinazione, l’entità TCP ricevente invia un segmento contrassegnato da un numero di
        acknowledgement uguale al numero di sequenza successivo che prevede di ricevere. Se il timer
        del mittente scade prima della ricezione dell’acknowledgement, il mittente ritrasmette il
        segmento.\\

        In sunto, le caratteristiche importanti da ricordare che TCP offre sono le seguenti:
        \begin{itemize}
            \item \textbf{Trasferimento bufferizzato}: il trasferimento viene ottimizzato creando pacchetti di
            dimensione il più possibile simile, accorpando o suddividendo in segmenti;
            \item \textbf{Orientamento dello stream}: i dati vengono passati al livello applicazione in ordine;
            \item \textbf{Stream non strutturato}: il servizio di stream del TCP non rispetta eventuali strutture
            presenti in dati strutturati: la comprensione della forma sta all’applicazione;
            \item \textbf{Connessione full-duplex}: trasferimento simultaneo in entrambe le direzioni;
            \item \textbf{Connessione connection oriented}: solo quando mittente e destinatario hanno
            verificato la sussistenza delle condizioni necessarie, ha inizio il trasferimento;
            \item \textbf{Riscontro positivo con trasmissione} (acknowledgement): alla ricezione di un
            pacchetto, il destinatario risponde con un ACK, ossia una conferma di ricezione.
            Eventualmente il mittente ritrasmette i pacchetti persi;
            \item \textbf{Sliding Window}: permette una trasmissione ottimizzata, seppure sia comunque più
            lenta rispetto ad UDP.
        \end{itemize}