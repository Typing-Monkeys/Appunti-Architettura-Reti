\section{Posta Elettronica}
La \textbf{posta elettronica} (o \textbf{e-mail}, dall’inglese \textbf{electronic mail}) è un servizio Internet che
permette lo scambio di messaggi tra utenti.\\

I sistemi di posta elettronica sono normalmente composti da 2 software:
\begin{itemize}
    \item \textbf{MUA (Mail User Agent, agente utente)}: consente alle persone di leggere e inviare la
posta elettronica (es: gmail, outlook…).
Costituisce un’interfaccia utente con le seguenti funzionalità:
    \begin{itemize}
        \item \textbf{Composizione}: processo di creazione di messaggi e risposte, che comprende
        assistenza per l’indirizzario dei corrispondenti e i numerosi campi di
        intestazione associati a ogni messaggio. Si pensi ai casi in cui si sceglie di
        rispondere ad una mail e vengono predisposti già alcuni campi;
        \item \textbf{Visualizzazione}: lettura dei messaggi in arrivo, che comprende fasi preliminari
        di conversione per gli allegati e strumenti per l’archiviazione dei messaggi;
        \item \textbf{Collocazione}: operazione svolta dal destinatario con il messaggio dopo la
        ricezione. Il messaggio può essere cancellato, recuperato, riletto, inoltrato o
        elaborato in altri modi.
    \end{itemize}

    \item \textbf{MTA (Mail Transfer Agent, agente di trasferimento)}: sposta i messaggi dall’origine
    alla destinazione (es: sendmail).
    Corrisponde al “postino” della posta reale, con le seguenti funzionalità:
    \item \textbf{Trasferimento}: spostamento dei messaggi dal mittente al destinatario. Avviene
    tramite l’attivazione di una sessione con il server di destinazione o qualche
    macchina intermedia qualora il server sia congestionato o irraggiungibile e sia
    disponibile il servizio di \textbf{\textit{mail relay}}, un gestore di posta secondario, che in caso
    di necessità sia in grado di salvare temporaneamente i messaggi in transito,
    facendo le veci del server di destinazione. Siccome veniva sfruttata dagli
    spammer, che si fingevano server secondari, adesso si indicano espressamente
    gli host dai quali si accetta relaying;
    \item \textbf{Reporting}: comunicazione al mittente di ciò che è avvenuto al messaggio, ad
    esempio se è stato consegnato, rifiutato, perso o si è presentato qualche errore.
\end{itemize}

Nel caso di sendmail, la configurazione manuale (da scrivere nel file
\textbf{/etc/mail/sendmail.cf}), è diventata nel tempo complessa, ed è stata semplificata
mediante l’uso di uno pseudolinguaggio denominato \textbf{M4}, che consente di attivare delle
funzioni mediante invocazione di macro. Alcuni file che riguardano sendmail sono:
\begin{itemize}
    \item \textbf{/etc/mail/access}: domini o host per i quali si controlla l’accesso;
    \item \textbf{/etc/mail/relay-domains}: domini per i quali si accetta relaying;
    \item \textbf{/etc/mail/sendmail.cw}: domini per i quali si fa virtual hosting;
    \item \textbf{/etc/aliases}: alias per indirizzi di posta elettronica.
\end{itemize}

La maggior parte dei sistemi di posta consente di creare \textbf{mailbox (caselle di posta)} per
archiviare la posta in arrivo di singoli utenti. Talvolta però, può risultare utile inviare un
messaggio a più persone correlate tra loro. Da qui nasce l’idea delle \textbf{mailing list}, elenchi di
posta elettronica, in cui quando viene inviato un messaggio, copie identiche vengono
consegnate a tutti i membri dell’elenco. Tutti gli indirizzi di posta elettronica sono del tipo:
\begin{center}
    \textbf{utente@indirizzo-dns}
\end{center}
Dove indirizzo-dns è un dominio DNS (che può contenere sottodomini).
    subsection{RFC822}
    RFC822 definisce una serie di campi di intestazione per le e-mail. Ogni campo consiste in una
    singola riga di testo ASCII contenente il nome del campo, un carattere di due punti e, per la
    maggior parte dei campi, un valore.\\
    Di seguito vengono elencati i campi di intestazione principali:
    %inserire tabella 1 pagina 44%

        \subsubsection{MIME}
        Per poter far fronte alle nuove esigenze di codifica di caratteri speciali (es: accenti, alfabeti non
        latini…) e contenuti multimediali (es: audio, immagini…), gli RFC1341 e 1521 hanno introdotto
        lo standard di codifica \textbf{MIME (Multipurpose Internet Mail Extensions)}. Esso aggiunge nuovi
        campi di intestazione all’RFC822, senza influire sul normale funzionamento del MTA.\\
        Di seguito alcuni dei più importanti:
        %inserire tabella 2 pagina 44%

        Per il Content-Transfer-Encoding di seguito sono elencati i principali schemi di codifica:
        \begin{itemize}
            \item \textbf{ASCII}: la codifica più diffusa, rappresenta i caratteri in 7 bit e per ciascuna riga del
            messaggio non si può eccedere i 1000 caratteri;
            \item \textbf{EBCDIC}: come ASCII e con la sua stessa limitazione, ma rappresenta i caratteri in 8 bit;
            \item \textbf{Base64}: a volte chiamato armatura ASCII, è utilizzata spesso per rappresentare i file
            binari che violano il limite di 1000 caratteri per riga. Vengono fatti gruppi di 24 bit, a
            loro volta suddivisi in 4 unità di 6 bit, che vengono inviate come carattere ASCII legale.
            Le sequenze == e = indicano rispettivamente che l’ultimo gruppo contiene 8 o 16 bit;
            \item \textbf{Quoted-printable}: per i messaggi quasi interamente ASCII ma contenenti alcuni
            messaggi non ASCII, la codifica base64 è inefficiente e dal suo posto si utilizza questa. Si
            tratta di ASCII a 7 bit, con tutti i caratteri superiori a 127 codificati come segni uguale
            seguiti dal carattere espresso da due cifre esadecimali.
        \end{itemize}
        
        Il \textit{Content-Type} è costituito da 7 tipi, ognuno con un certo numero di sottotipi, ed è nella forma:
        \begin{center}
            \textbf{Content-Type: tipo/sottotipo}
        \end{center}
        I 7 tipi sono: \underline{Text, Image, Audio, Video, Application, Message, Multipart.}

    \subsection{Il trasferimento dei messaggi: SMTP}
    
    \textbf{SMTP (Simple Mail Transfer Protocol)} è un protocollo adibito al trasporto di messaggi
    efficiente ed affidabile, in ambienti eterogenei che coinvolgono un client ed un server (dove il
    server può essere il destinatario finale o un server intermedio) tramite la realizzazione di un
    \textbf{IPCE (InterProcess Comunication Environment)}, ossia la realizzazione di un circuito
    virtuale tra server e client che consente la consegna del messaggio.
    Come prima cosa, il client stabilisce una connessione \textbf{TCP} con la \textbf{porta 25} del server ed attende
    una risposta. Il server invia quindi una riga di testo che comunica la sua identità e la possibilità
    di inviare la posta, con conseguente attivazione del canale trasmissivo. Se ciò non avviene, il
    client rilascia la connessione e riprova in seguito.\\

    Una volta attivato il canale, il primo comando obbligatorio che il client deve inviare è \textbf{HELO},
    necessario per inizializzare una sessione SMTP ed identificarsi. Se l’argomento non è
    accettabile, il server ritorna un errore 501 Failure.\\
    
    In seguito, il client può inviare una sequenza di comandi, a passi bloccanti. Per l’invio di
    messaggi, si usano in sequenza:
    \begin{itemize}
        \item \textbf{MAIL FROM}: viene passato l’indirizzo del mittente;
        \item \textbf{RCPT TO}: viene passato l’indirizzo del destinatario. Può essere ripetuto per specificare
        più destinatari;
        \item \textbf{DATA}: vengono passati i dati del messaggio. Al termine dell’immissione del testo, deve
        seguire la sequenza \textbf{<CR><LF>.<CR><LF>}, dove <CR><LF> è una serie di caratteri che
        viene interpretato come “vai a capo”.
    \end{itemize}
    
    La transazione può essere abortita in qualsiasi momento, con il comando \textbf{RSET}. Ogni qualvolta
    il server interpreta correttamente un messaggio, risponde con \textbf{OK}. Altri comandi utilizzabili
    ovunque durante una sessione sono NOOP, HELP, EXPN e VRFY. L’ultimo comando di una
    sessione è \textbf{QUIT}.
    
    \subsection{POP3}

    \textbf{POP3 (Post Office Protocol version 3)}, noto anche come \textbf{Popper}, è un protocollo client-server
    che permette l’accesso da remoto ad un server di posta elettronica. Una volta connesso al
    server, è possibile recuperare la posta (conservandola in locale) e/o cancellare i messaggi dal
    server SMTP. Si noti che l’invio della posta resta comunque affidato a SMTP.\\

    Esso viene avviato quando l’utente apre il lettore della posta. Il lettore della posta chiama l’ISP
    (a meno che sia già disponibile una connessione) e stabilisce una connessione TCP con il MTA
    sulla porta 110. Una volta stabilita la connessione, POP3 attraversa sequenzialmente 3 stati:\\
    \begin{itemize}
        \item \textbf{Authorization}: login dell’utente (si noti che le credenziali vengono passate in chiaro),
        effettuato tramite i comandi \textbf{USER} e \textbf{PASS};
        \item \textbf{Transaction}: ottenimento o cancellazione della posta dalla casella dell’ISP. I comandi
        utilizzati sono:
        \begin{itemize}
            \item \textbf{STAT}: ritorna numero dei messaggi nella casella e dimensione in byte;
            \item \textbf{LIST} [msg]: dato un msg (ID del messaggio), ritorna ID e dimensione. Se non si
            passa alcun ID, elenca tutti i messaggi (con ID e dimensione di ognuno);
            \item \textbf{RETR} msg: ricezione del messaggio specificato;
            \item \textbf{DELE} msg: cancellazione del messaggio specificato;
            \item \textbf{NOOP}: No Operation, si usa per mantenere la connessione aperta;
            \item \textbf{LAST}: ritorna l’ID dell’ultimo messaggio ricevuto;
            \item \textbf{RSET}: resetta la connessione al suo stato iniziale, smarcando tutti i messaggi
            marcati per l’eliminazione.
        \end{itemize}
        \item \textbf{Update}: provoca l’effettiva eliminazione della posta, terminando la sessione. Si effettua
        tramite il comando \textbf{QUIT}.
    \end{itemize}
    
    Il server risponde alle richieste con esito positivo (\textbf{+OK}) oppure con esito negativo (\textbf{-ERR}).

    \subsection{IMAP}
    Uno svantaggio di POP3 è che i messaggi devono essere scaricati sul client per poter essere letti.
    Si pensi al caso in cui si vuole accedere alla stessa casella di posta elettronica da macchine
    diverse (smartphone, portatile, computer fisso…): utilizzando POP3, si ottiene come risultato
    che la posta dell’utente viene rapidamente dispersa su più macchine.\\

    Questo svantaggio ha portato alla nascita di un protocollo di consegna alternativo, \textbf{IMAP
    (Internet Message Access Protocol)}, un protocollo definito in RFC2060 che utilizza la porta
    143. A differenza di POP3, che presume l’utente cancelli la casella di posta ogni volta, IMAP
    presume che tutti i messaggi rimangano sul server per un tempo indeterminato.\\

    Le differenze e migliorie con POP3 non sono poche e di seguito se ne vedono alcune:
    \begin{itemize}
        \item \textbf{Decine di nuovi comandi e opzioni} (molta più flessibilità e personalizzazione);
        \item \textbf{Leggere in tutto o in parte i messaggi} (il che ad esempio è utile mentre si utilizza un
        modem lento, per leggere il testo di un messaggio con grandi allegati audio e video);
        \item \textbf{Effettuare ricerche indicizzate} (senza il bisogno di scaricare tutte le mail);
        \item \textbf{Sincronizzazione} tra i vari client (nel caso si acceda da più client simultaneamente);
        \item \textbf{Supporto per attributi dei messaggi} (per esempio per sapere se un messaggio è già
        stato letto o se ha avuto una risposta).
    \end{itemize}
    

    \subsection{Posta Elettronica Privata}
    \textit{L’argomento è stato trattato in maniera sintetica.}\\
        
    Anche per la posta elettronica può essere necessario garantire la segretezza dei messaggi, ma
    nativamente non ci sono meccanismi che possono garantirla. È risaputo che crittografia e
    certificati digitali permettono di risolvere questo problema, e sono diversi i software che sono
    stati creati per implementarli nelle mail.

        \subsubsection{PGP}
        \textbf{PGP (Pretty Good Privacy)} è una famiglia di software multipiattaforma di crittografia
        utilizzati per autenticazione e privacy. Nell’ambito della posta elettronica, consente la
        crittografazione, la firma digitale e la compressione dei messaggi ed è attualmente utilizzato da
        Enigmail e Mozilla Thunderbird. Utilizza 3 algoritmi di crittografazione durante le sue
        operazioni: \textbf{RSA, IDEA} e \textbf{MD5}.\\

        Un’alternativa Open Source è \textbf{GnuPG}, pensato come rimpiazzato di PGP. Utilizza un metodo per
        la condivisione efficiente delle chiavi pubbliche, usando una rete particolare: \textbf{Web of Trust}.
