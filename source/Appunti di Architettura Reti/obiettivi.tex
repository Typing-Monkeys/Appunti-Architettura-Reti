
\section*{Obiettivi}
    L’obiettivo principale dei presenti appunti è la preparazione personale alla disciplina
    “Architettura Reti” insegnata dal prof. Osvaldo Gervasi dell’Università di Perugia.\newline

    Non conoscendo nulla riguardo l’architettura di una rete ed i suoi protocolli, il materiale reso
    disponibile dal docente (slides) si è rivelato insufficiente per un ripasso o uno studio nei casi in
    cui non sia stato presente a lezione. Anche altri appunti resi gentilmente disponibili da altri
    studenti non si sono rivelati chiari in molti punti.\newline\newline
    Per cui, l’obiettivo di questi appunti è fornire una buona visione d’insieme di ogni argomento
    presente nel programma 2019/20 di Architettura Reti, tenendo a mente che il lettore potrebbe
    non aver mai letto nulla riguardo la suddetta materia e fornendo quindi anche definizioni
    basilari e spiegazioni base.\newline

    Ove opportuno, alcuni argomenti sono presentati in forma anche leggermente più estesa
    solamente a scopo di chiarimento. Tuttavia, altri argomenti sono particolarmente dettagliati e
    non tutti i dettagli sono ovviamente affrontati durante le lezioni. In questi casi, si è deciso di
    limitarsi ad una visione d’insieme più che sufficiente per la preparazione all’esame, ma che
    potrebbe non risultare esaustiva per gli studenti più curiosi. Questi argomenti sono
    singolarmente segnalati in ogni paragrafo qualora si volesse approfondire.\newline\newline
    Inoltre, è bene segnalare che non viene trattato l’intero programma dell’anno 2019/2020, in
    quanto sono stati tralasciati i seguenti argomenti:
    
    \begin{itemize}
        \item Topologia di reti;
        \item Livello fisico/data link OSI (mezzi trasmissivi, ethernet, metodi di accesso al bus ...);
        \item Frame Relay e ATM;
        \item VPN.
    \end{itemize}

    Si noti che è comunque possibile contribuire alla repository dedicata su GitHub
    (\url{https://github.com/CoffeeStraw/Appunti-Architettura-Reti}) per ultimare e tenere aggiornati
    i presenti appunti, modificando il file .docx con cui sono stati scritti e rigenerando il .pdf
    utilizzato per la distribuzione.