\section{Protocolli di controllo Internet}
    Oltre a IP, utilizzato per il trasferimento dei dati, Internet ha diversi protocolli di controllo
    utilizzati nel livello network, tra cui \textbf{ICMP}, \textbf{ARP} e \textbf{RARP}.

    \subsection{ICMP}
        Il funzionamento di Internet è monitorato attentamente dai router. Quando avviene qualcosa
        di imprevisto, l’evento è comunicato da \textbf{ICMP (Internet Control Message Protocol)}. Svolge
        inoltre una seconda funzione, che è quella di verificare lo stato della rete. Di seguito i principali
        tipi di messaggi ICMP.

        %TABELLA ICMP PAGINA 24%

    \subsection{ARP}
        Un host Internet può comunicare con un altro host solo se conosce il suo indirizzo fisico di rete
        (MAC), che è anche fondamentale per il funzionamento del gateway al fine di definire il
        percorso dei pacchetti. Attenzione però: i programmi applicativi in genere conoscono solo il
        nome dell’host o il suo indirizzo IP.\\
        
        \textbf{ARP (Address Resolution Protocol)} fornisce il modo con cui gli indirizzi IP vengono associati
        agli indirizzi del livello data link (per esempio agli indirizzi Ethernet), risolvendo la
        corrispondenza Indirizzo IP-Indirizzo fisico.\\

        L’host che ha bisogno dell’indirizzo fisico di un altro host della rete, invia un pacchetto
        \textbf{broadcast} di richiesta dell’indirizzo al destinatario (includendo il proprio indirizzo fisico di
        rete), il quale risponde includendo a sua volta l’indirizzo fisico di rete.\\

        In ciascuna macchina è predisposta una cache, che memorizza gli indirizzi richiesti via ARP.
        Questo risulta utile per velocizzare le consultazioni successive. In particolare, ad ogni richiesta
        ARP tutti gli host possono aggiornare l’indirizzo fisico del richiedente nella propria cache.
        
    \subsection{RARP}
        Come detto, ARP risolve il problema della scoperta dell’indirizzo Ethernet che corrisponde a un
        dato indirizzo IP. In qualche caso è utile però risolvere il problema inverso: dato un indirizzo
        Ethernet, qual è il corrispondente indirizzo IP? Un caso in cui è molto utile risolvere questo
        problema, è l’avvio delle workstation diskless, che hanno bisogno di richiedere da un file server
        remoto un’immagine binaria del sistema operativo. In questo caso, come fanno a scoprire il
        proprio indirizzo IP?\\
        
        Ecco quindi che interviene \textbf{RARP (Reverse Address Resolution Protocol)}. RARP effettua una
        richiesta \textbf{broadcast} contenente il proprio indirizzo Ethernet, in cui domanda se qualcuno
        conosce il suo indirizzo IP. I server che ricevono la richiesta, cercano l’indirizzo Ethernet nei
        propri file di configurazione e trasmettono una risposta contenente l’indirizzo IP
        corrispondente.\\

        Se i riceventi non sanno rispondere, propagano la richiesta ai server secondari.
        